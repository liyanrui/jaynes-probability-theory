\environment env-mingyi

\starttext

\input cover

\setuppagenumbering[conversion=numbers]
\setuppagenumber[number=1]
\setups{Text}

\chapter[plausible-reasoning]{可信推理}

\chapter[quatitative-rules]{量化规则}

\startdictum
概率论,不过是生活约化而成的计算。
  
\rightaligned{拉普拉斯,1819}
\stopdictum

\indentation 问题已被形式化了,这是我们所作的公设在数学上的必然结果。这些公设可粗略描述如下:

\startitemize[R]
\item 命题的可信性由实数表示;
\item 定性符合常识;
\item 一致性。
\stopitemize

\noindent 本章仅基于这些公设推导与推断相关的量化规则。所得结果,曾经有着一段漫长、复杂又令人难以置信的历史。这段历史充满着广义科学方法论方面的教训(见某些章尾部的评注部分)。

\section{乘法规则}

我们首先为逻辑乘 $AB$ 的可信性分别与 $A$ 与 $B$ 的可信性之间的关系找出一个一致性的规则。我们在意的其实是 $AB|C$。由于推理过程本身有些微妙,这需要我们从一些不同的视角去审视。

先考虑将 $AB$ 为真的裁决打碎为对 $A$ 与 $B$ 所作的更为基本的裁决。机器人可以

\startitemize[n]
\item 裁决 $B$ 为真;\hfill $(B|C)$
\item 认可 $B$ 为真,裁决 $A$ 为真。\hfill $(A|BC)$
\stopitemize

\noindent 或者,等同地,

\startitemize[n]
\item 裁决 $A$ 为真;\hfill $(A|C)$
\item 认可 $A$ 为真,裁决 $B$ 为真。\hfill $(B|AC)$
\stopitemize

\noindent 上面,在每种情况中,我们给出了每一步骤相应的可信性。

现在,解释一下第 1 个过程。为了命题 $AB$ 为真,$B$ 必须为真。因而,必须考虑 $B|C$ 的可信性。此外,若 $B$ 为真,则需要进一步保证 $A$ 为真,因此还需要考虑 $A|BC$ 的可信性。但是若 $B$ 为假,则无需裁决 $A$,亦即无需裁决 $A|\itbar{B}C$,便可知晓 $AB$ 为假;若机器人先推理 $B$,那么,$A$ 的可信性仅在 $B$ 为真时才值得考虑。因此,机器人在有了 $B|C$ 与 $A|BC$ 的情况下,就不需要 $A|C$ 了,因为 $A|C$ 并未给它能带来更多的信息。

类似地,$A|B$ 与 $B|A$ 是不需要的;无论 $A$ 与 $B$ 在缺乏 $C$ 的情况下多么可信,都与机器人在知道 $C$ 为真的情况下所作的裁决无关。例如,若机器人知晓地球是圆的,那么在裁决与现在的宇宙学相关的问题时,它就可以忽略那些在它尚不知地球是圆的之时需要考虑的观点。

由于逻辑乘法运算遵循交换律,$AB = BA$,因此上文中的 $A$ 与 $B$ 毫无疑问,可以互换;亦即 $A|C$ 与 $B|AC$ 的知识也能用于确定 $AB|C = BA|C$。对于 $AB$,机器人必定能从这两个过程中得到相同的可信性,这是我们的一致性条件之一,即公设 (\convertnumber{R}{3}a)。

可以用更明确的形式来描述。$(AB|C)$ 可以是 $B|C$ 与 $A|BC$ 的某个函数:

\placeformula[basic-function]
\startformula
(AB|C) = F[(B|C), (A|BC)]
\stopformula

现在,如果上述推理尚不完全清晰,那么就审视一下其他的替代形式。例如,可以假设

\placeformula
\startformula
(AB|C) = F[(A|C), (B|C)]
\stopformula

是容许的形式。但是,很容易揭示,这种形式的关系无法满足公设 (\convertnumber{R}{2}) 的那些定性条件。给定 C,命题 $A$ 或许非常可信,命题 $B$ 或许可信,但是 $AB$ 可能依然非常可信或非常不可信。

例如,下一个遇到的人,蓝眼睛,这相当可信,黑头发,这也相当可信;这两样生理特征同时出现于此人身上,也没什么不合理之处。不过,左眼睛是蓝色的,这相当可信,右眼睛是褐色的,也相当可信,但是若它们同时为真,那就极为不可信。如果使用这种形式的公式,那么就没法考虑这些情况。用这种形式的函数,我们的机器人无法像人类那样去作推理,它甚至连定性的推理都做不到。

还有一些其他的可能的形式。尝试所有的可能形式的方法——\quotation{穷举证明}——可像下面这样进行。先引入一些实数

\placeformula
\startformula
u = (AB|C),\quad v = (A|C),\quad w = (B|AC),\quad x = (B|C),\quad y = (A|BC).
\stopformula

如果 $u$ 被表示成 $u$、$v$、$x$、$y$ 中两个或更多个实数的函数,那么有 11 种可能的形式。可以写出每一种可能,将它置于各种极端条件下,如同褐色眼睛与蓝色眼睛那样(抽象描述:$A$ 蕴含了 $B$ 为假\footnote{译注:$A\Rightarrow\itbar{B}$})。其他极端条件有,$A = B$,$A = C$,$C\Rightarrow\itbar{A}$,等。Tribus(1969)作了冗长乏味的分析,发现只有两种可能的形式,在某种极端的情况下,遵循着定性符合常识这一公设,它们分别是 $u = F(x, y)$ 与 $u = F(w, v)$,这正是之前我们推理出来的那两种函数形式。

现在,运用第 \in[plausible-reasoning] 章讨论的量化要求。给定的先验信息的任何变化 $C\rightarrow C'$,以及 $B$ 变得更可信,$A$ 没有变化,

\placeformula
\startformula
B|C' > B|C
\stopformula

\placeformula
\startformula
A|BC' = A|BC
\stopformula

根据常识,$AB$ 只会变得更可信,而不是更不可新:

\placeformula
\startformula
AB|C'\ge AB|C
\stopformula

当且仅当 $A|BC$ 为不可能时,相等关系成立。同样,给定先验信息 $C''$,以及

\placeformula
\startformula
B|C'' = B|C
\stopformula

\placeformula
\startformula
A|BC'' > A|BC
\stopformula

就会有

\placeformula
\startformula
AB|C'' \ge AB|C
\stopformula

当且仅当 $B|C$ 为不可能时,相等关系成立(即使此时未去定义 $A|BC$,$AB|C''$ 依然不可能为真)。此外,函数 $F(x, y)$ 必须是连续的,不然,(\in[basic-function]) 右侧的某个可信性的微小递增可能会导致 $AB|C$ 的大幅递增。

总之,$F(x, y)$ 必须是 x 与 y 的连续的单调递增函数。若假设它可微(并非必须如此;见 (\in[function-eq]) 的讨论),便有

\startsubformulas[mono-req]
\placeformula
\startformula
F_1(x, y) \equiv \frac{\partial F}{\partial x} \ge 0
\stopformula

仅当 $x$ 表示不可能时,上式中的相等关系成立;还有,

\placeformula
\startformula
F_2(x, y) \equiv \frac{\partial F}{\partial y} \ge 0
\stopformula
\stopsubformulas


后文会继续使用这些符号,无论 $F$ 是什么样的函数,$F_i$ 表示 $F$ 的第 $i$ 个参数对应的微分。

接下来,我们动用公设 (\convertnumber{R}{3}a),\quotation{结构上}的一致性。假设想获得 $(ABC|D)$ 的可信性,即三个命题同时为真的可信性,由于布尔运算遵循结合律 $ABC = (AB)C = A(BC)$,因此有两种方式来做此事。若规则具有一致性,一定能从这两种顺序的运算中获得相同的结果。首先,可以将 $BC$ 视为单一的命题,利用 (\in[basic-function]):


\placeformula
\startformula
(ABC|D) = F[(BC|D), (A|BCD)]
\stopformula

然后,对可信性 $(BC|D)$ 应用 (\in[basic-function]),可得:

\startsubformulas
\placeformula[subeq:1]
\startformula
(ABC|D) = F\{F[(C|D), (B|CD)], (A|BCD)\}
\stopformula

不过,也可以在一开始将 $AB$ 视为单一的命题,这样就可以从另一种顺序推出不同的结果:

\placeformula[subeq:2]
\startformula
(ABC|D) = F[(C|D), (AB|CD)] = F\{(C|D), F[(B|CD), (A|BCD)]\}
\stopformula
\stopsubformulas

若我们寻找的这个规则是要用于表示推理的一致性方法,那么 (\in[subeq:1]) 与 (\in[subeq:2]) 必定恒等。在这种情况下,我们的机器人所作的一致性推理,其必要条件可表示为一个函数方程,

\placeformula[function-eq]
\startformula
F[F(x, y), z] = F[x, F(y, z)]
\stopformula

在数学里,这个方程历史悠久,源起于 N. H. Abel (1826) 的研究。Acz\'el (1966) 在他的讲述函数方程的巨著中,贴切地成这个方程为\quotation{关联方程}(Associativity Equation),并列举了 98 份讨论或使用这个方程的参考文献。Acz\'el 在不假设函数可微的条件下,求出了通解 (\in[aczel-general-solution]),见下文。不过,他的书中(Acz\'el, 1987),用了 11 页的篇幅证明此解的存在。在此,我们给出 R. T. Cox (1961) 在假设函数可微的前提下给出的更短的证明;也可参考附录 B 中的讨论。

显然,(\in[function-eq]) 有一个平凡解,$F(x, y) = \text{常数}$。但是这个解违背了单调性要求 (\in[mono-req]),并且毫无用处可言。除非 (\in[function-eq]) 有一个非平凡解,不然,这条路就走不通了;因此,必须寻求最为广义的非平凡解。先定义一些缩写

\placeformula
\startformula
u\equiv F(x, y),\quad v\equiv F(y, z)
\stopformula

现在,依然认为 $(x, y, z)$ 是互不依赖的变量,待求解的函数方程可写为

\placeformula
\startformula
F(x, v) = F(u, z)
\stopformula

分别对 $x$ 与 $y$ 求微分,按照 (\in[mono-req]) 的记法,可得

\placeformula
\startformula
\startmathalignment
\NC F_1(x, v) \NC = F_1(u, z)F_1(x, y)\NR
\NC F_2(x, v)F_1(y, z) \NC = F_1(u, z)F_2(x, y)\NR
\stopmathalignment
\stopformula

从这两个方程中消除 $F_1(u, z)$,可得

\placeformula
\startformula
G(x, v)F_1(y, z) = G(x, y)
\stopformula

\stoptext