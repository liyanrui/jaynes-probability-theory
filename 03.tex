\chapter[quantitative-rules]{初等采样理论}

至此,我们所拥有并且有效的数学基础由基本的乘法和加法规则构成:

\placeformula[3-1]
\startformula
P(AB|C) = P(A|BC)P(B|C) = P(B|AC)P(A|C)
\stopformula

\placeformula[3-2]
\startformula
P(A|B) + P(\itbar{A}|B) = 1
\stopformula

从它们可以导出扩展的加法规则:

\placeformula[3-3]
\startformula
P(A + B|C) = P(A|C) + P(B|C) - P(AB|C)
\stopformula

使用一致性的祈求 (\Roman{3}c),亦即无差异性原理:如果一组假设 $(H_1,H_2,\cdots,H_N)$ 在背景信息 $B$ 上相互独立且详尽,并且 $B$ 不会偏好这些假设中的任何一个,那么

\placeformula[3-4]
\startformula
P(H_i|B) = \frac{1}{N}\quad\quad 1\le i \le N
\stopformula

从 (\in[3-3]) 和 (\in[3-4]) 又可以推导出 Bernoulli 瓮规则:如果 $B$ 指定了 $A$ 在 $M$ 个假设所构成的子集上为真,而在剩下的 $(N - M)$ 假设所构成的子集为假,那么

\placeformula[3-5]
\startformula
P(A|B) = \frac{M}{N}
\stopformula

意识到概率论在内容上有多少能够仅仅从此式推导出来,这一点非常重要。

实际上,当前所教授的概率论的全部内容,加上许多经常重要结果——它们经常被视为超出了概率论范围,可以从上述基础推导出来。接下来的几章内容会给出一些细节,而后在第 11 章,继续发展我们的机器人的大脑,让它能够充分理解高级应用所需要的另外一些原理。

在本章中,我们的概率论,它的第一个应用与我们之后所期望达到的严肃科学推断相比,相当简单和幼稚。无论如何,我们从细节上来考虑这些,原因并不仅仅是面向教学。失于理解这些最为简单的应用的逻辑已经是几十年来阻碍科学推断发展进程——也包含科学本身——的主要原因之一。因而我们强烈建议读者,即使你早已熟悉初等采样理论,在处理更为复杂的问题之前,也要认真消化本章内容。

\section{无放回采样}

为了让 Bernoulli 瓮更为明确,我们定义了以下命题。

\startnarrower[2*middle]
$B\equiv$ 有个瓮,里面有 $N$ 个球。它们的编号是 $(1,2,\cdots,N)$,其中有 $M$ 个球是红色的,其余的是白色的。除此之外,这些球的各方面都相同。
\stopnarrower