\chapter[quantitative-rules]{初等采样理论}

至此,我们所拥有并且有效的数学基础由基本的乘法和加法规则构成:

\placeformula[3-1]
\startformula
P(AB|C) = P(A|BC)P(B|C) = P(B|AC)P(A|C)
\stopformula

\placeformula[3-2]
\startformula
P(A|B) + P(\itbar{A}|B) = 1
\stopformula

从它们可以导出扩展的加法规则:

\placeformula[3-3]
\startformula
P(A + B|C) = P(A|C) + P(B|C) - P(AB|C)
\stopformula

使用一致性的祈求 (\Roman{3}c),亦即无差异性原理:如果一组假设 $(H_1,H_2,\cdots,H_N)$ 在背景信息 $B$ 上相互独立且详尽,并且 $B$ 不会偏好这些假设中的任何一个,那么

\placeformula[3-4]
\startformula
P(H_i|B) = \frac{1}{N}\quad\quad 1\le i \le N
\stopformula

从 (\in[3-3]) 和 (\in[3-4]) 又可以推导出 Bernoulli 瓮规则:如果 $B$ 指定了 $A$ 在 $M$ 个假设所构成的子集上为真,而在剩下的 $(N - M)$ 假设所构成的子集为假,那么

\placeformula[3-5]
\startformula
P(A|B) = \frac{M}{N}
\stopformula

意识到概率论在内容上有多少能够仅仅从此式推导出来,这一点非常重要。

实际上,当前所教授的概率论的全部内容,加上许多经常重要结果——它们经常被视为超出了概率论范围,可以从上述基础推导出来。接下来的几章内容会给出一些细节,而后在第 11 章,继续发展我们的机器人的大脑,让它能够充分理解高级应用所需要的另外一些原理。

在本章中,我们的概率论,它的第一个应用与我们之后所期望达到的严肃科学推断相比,相当简单和幼稚。无论如何,我们从细节上来考虑这些,原因并不仅仅是面向教学。失于理解这些最为简单的应用的逻辑已经是几十年来阻碍科学推断发展进程——也包含科学本身——的主要原因之一。因而我们强烈建议读者,即使你早已熟悉初等采样理论,在处理更为复杂的问题之前,也要认真消化本章内容。

\section{无放回采样}

为了让 Bernoulli 瓮更为明确,我们定义了以下命题:

\definedescription[mydefinition][location=left, before=, after=, headstyle=\tf, width=broad, distance=0.25em]
\def\myproposition#1{\hbox to 1.5em{#1}$\equiv$}
\mydefinition{\myproposition{$B$}} 有个瓮,里面有 $N$ 个球。它们的编号是 $(1,2,\cdots,N)$,其中有 $M$ 个球是红色的,其余的是白色的,$0\le M \le N$。除此之外,这些球的各方面都相同。闭着眼从瓮中取一个球,观察并记录它的颜色,再把它放回去。重复这一过程 $n$ 次,$0\le n \le N$。\par
\mydefinition{\myproposition{$R_i$}} 第 $i$ 次取的球为红色。\par
\mydefinition{\myproposition{$W_i$}} 第 $i$ 次取的球为白色。\par

根据 $B$,能够取的球仅有红色和白色,有

\placeformula[3-6]
\startformula
P(R_i|B) + P(W_i|B) = 1\quad\quad 1\le i \le N
\stopformula

它等同于说,在知识 $B$ 所创立的\quotation{逻辑环境}里,命题 $R_i$ 与 $W_i$ 的关系是互反的,即

\placeformula[3-7]
\startformula
\itbar{R_i} = W_i\quad\quad \itbar{W_i} = R_i
\stopformula

并且,对于第一次取球,(\in[3-5]) 变为

\placeformula[3-8]
\startformula
P(R_i|B) = \frac{M}{N}
\stopformula

\placeformula[3-9]
\startformula
P(W_i|B) = 1 - \frac{M}{N}
\stopformula

现在来看这意味着什么。概率赋值 (\in[3-8]) 和 (\in[3-9]) 所断言的并非瓮的物理属性或其内容;它们描述的是机器人在取球之前所具备的{\bf 知识状态}。实际上,倘若机器人的知识状态与上面定义的 $B$ 有所不同(例如,它知道红球和白球在瓮内的位置,或者它不知道 $N$ 和 $M$ 多大),那么它为 $R_i$ 和 $W_i$ 所赋的概率必然与 (\in[3-8]) 和 (\in[3-9]) 不同,而瓮的现实属性却不会变。

因此,称用瓮来做实验以\quotation{验证}(\in[3-8]),这种说法,如同在狗身上做实验来验证一个孩子对他的狗的喜爱,毫无逻辑可言。眼下,我们关心的是来自不完备信息的一致性推理所涉及的逻辑,而不是瓮中取出什么东西这一物理现象的断言(任何情况下,着都是不可能做到的,因为信息 $B$ 不完备)。

最终,我们的机器人将会能够作出一些非常肯定的现实预测。这些预测能够接近,但是(除非在一些退化的情况下)并不能真正达到逻辑演绎的确定性;但是,在我们能够说出什么量能够被准确预测并且为此需要何种信息之前,理论还需要进一步发展。换言之,由机器人在不同知识状态下所赋的概率与实验中可观察到的事实,这二者之间的关系可能并非随意建立的;