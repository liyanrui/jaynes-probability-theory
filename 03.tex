\chapter[quantitative-rules]{初等采样理论}

至此,我们所拥有并且有效的数学基础由基本的乘法和加法规则构成:

\placeformula[3-1]
\startformula
P(AB|C) = P(A|BC)P(B|C) = P(B|AC)P(A|C)
\stopformula

\placeformula[3-2]
\startformula
P(A|B) + P(\itbar{A}|B) = 1
\stopformula

从它们可以导出扩展的加法规则:

\placeformula[3-3]
\startformula
P(A + B|C) = P(A|C) + P(B|C) - P(AB|C)
\stopformula

使用一致性的祈求 (\Roman{3}c),亦即无差异性原理:如果一组假设 $(H_1,H_2,\cdots,H_N)$ 在背景信息 $B$ 上相互独立且详尽,并且 $B$ 不会偏好这些假设中的任何一个,那么

\placeformula[3-4]
\startformula
P(H_i|B) = \frac{1}{N}\quad\quad 1\le i \le N
\stopformula

从 (\in[3-3]) 和 (\in[3-4]) 又可以推导出 Bernoulli 瓮规则:如果 $B$ 指定了 $A$ 在 $M$ 个假设所构成的子集上为真,而在剩下的 $(N - M)$ 假设所构成的子集为假,那么

\placeformula[3-5]
\startformula
P(A|B) = \frac{M}{N}
\stopformula

意识到概率论在内容上有多少能够仅仅从此式推导出来,这一点非常重要。

实际上,当前所教授的概率论的全部内容,加上许多经常被认为是重要的结果