\startuseMPgraphic{crux}
color ccc;
pair p, h[], v[];
u := \overlaywidth; v := \overlayheight;
hdelta := .15u; vdelta := .05v;
drawoptions (withpen pensquare scaled 2pt);
randomseed := day + time*epsilon;
show time * epsilon;
for i := 1 upto 521:
    ccc := (uniformdeviate (1), uniformdeviate (1), uniformdeviate (1));
    p := (uniformdeviate (u), uniformdeviate (v));
    h0 := (xpart (p) - uniformdeviate (hdelta) , ypart (p));
    h1 := (xpart (p) + uniformdeviate (hdelta) , ypart (p));
    v0 := (xpart (p), ypart (p) - uniformdeviate (vdelta));
    v1 := (xpart (p), ypart (p) + uniformdeviate (vdelta));
    draw h0 -- h1 withcolor transparent(1,.4,ccc);
    draw v0 -- v1 withcolor transparent(1,.4,ccc);
endfor;
\stopuseMPgraphic

\startsetups BG
\defineoverlay[bg][\useMPgraphic{crux}]
\setupbackgrounds[page][background=bg]
\stopsetups
\definestartstop[BG][commands=\setups{BG}]

\startBG
\startstandardmakeup
\setuplayout[cover]
\placeintermezzo[none][]{}{
  \startframedtext
    [width=0.45\textwidth,
      frameoffset=10pt,
      rulethickness=4pt,
      framecolor=middlegreen,
      backgroundoffset=10pt,
      background=color,
      backgroundcolor=lightgray]
    [middle]
\switchtobodyfont[18pt]
{\bfd 概率论}
\blank[big]
\hfill{\bfb——科学逻辑}
\blank[big]
\hfill E. T. Jaynes\;\;著
\blank[small]
\hfill 李延瑞\;\;译
\stopframedtext
}
\vfill
\stopstandardmakeup
\stopBG