\chapter[plausible-reasoning]{可信推理}

\startdictum
逻辑学所关心的问题,要么是肯定或否定的,要么是完全不确定的。总之,(应当庆幸)没什么问题需要推理。世上真正的逻辑应该是概率计算——关乎可能性的量化运算,它存在甚至应该存在于理性者的思维之中。
  
\rightaligned{James Clerk Maxwell,1850}
\stopdictum

在漆黑的夜晚,一名警察在空无一人街道上巡逻。突然,他听到了警铃惊响,循声望去,街对面一家珠宝店的窗户洞开,一位蒙面的绅士,肩负一袋昂贵的珠宝,向外爬出。警察不假思索,立即断定此人绝非善类。他是基于什么得出这一结论的呢?

\section{演绎推理与可信推理}

警察的结论显然不是通过从证据出发的逻辑推导而得到。任何犯罪,皆能为之编造出完美的无罪辩护。譬如,那位绅士可能是珠宝店的主人。当天夜里,他从一个化妆舞会回家,发现没带钥匙。当他走到自己的店铺时,一辆驶过的卡车扔出一块石头击破了窗户。他为了保护自己的财产不得已而上演了警察所看到的一幕。

尽管警察的推理过程并非逻辑推导,但是我们得承认他的结论在一定程度上也是正确的。警察看到的那一幕,不能作为证据,去证明那位绅士犯罪,但是却能够让自己的结论看上去更为可信。这种推理方式或许是我们的本能。有些事,例如今天是否下雨,由于缺乏足够的信息,无法进行演绎推理,但是又必须当机立断。这样的事情在我们的清醒的时间里几乎无时无刻不在发生。

这种推理貌似寻常,其过程却非常微妙。尽管相关的讨论已经延续了 24 个世纪,但迄今依然没有令人满意的答案。在这本书里,我们会汇报一些有用且令人鼓舞的新的进展,即使用明确的定理去代替那些不靠谱的直觉判断,并且使用一些非常基本且近乎无法避免的理性标准所确定的规则代替那些特设的过程。

有关这些问题的所有讨论,始于几个例子,它们反映了演绎推理与可信推理的对立。演绎推理(证明式推理),亦即亚里士多德(公元前 4 世纪)\footnote{现在,对于亚里士多德的贡献的确切性有一些不同的看法。这方面的争议与我们开展的工作无关,但是感兴趣的读者可以从 Lukasiewicz(1957)那里找到这方面的一些讨论。} 的工具论,总是能够归结为两个强三段论的反复运用,即

\placeformula[syllogism-1]
\startformula
\syllogism{若 $A$ 为真,则 $B$ 为真}{$A$ 为真}{所以,$B$ 为真}
\stopformula

与

\placeformula[syllogism-2]
\startformula
\syllogism{若 $A$ 为真,则 $B$ 为真}{$B$ 为假}{所以,$A$ 为假}
\stopformula

这就是我们乐意时时使用的推理方式,但是很不幸,大部分情况下我们缺乏这种推理所需要的信息,只好退而求其次,使用弱三段论:

\placeformula[weak-1]
\startformula
\syllogism{若 $A$ 为真,则 $B$ 为真}{$B$ 为真}{所以,$A$ 更可信}
\stopformula

证据不足以证明 $A$ 为真,但是若 $A$ 的结果之一得到确证,我们就会对 $A$ 为真更有信心。例如,设

\startformula
\startalign
\NC A\equiv\NC\text{\bf 最晚上午十点下雨。}\NR
\NC B\equiv\NC\text{\bf 上午十点之前多云。}\NR
\stopalign
\stopformula

在上午 9:45 看到云,并不能断定将会下雨。我们的常识是服从弱三段论的,它会督促我们改变计划与行动,因此如果那些云足够黑,我们就会相信会下雨。

这个例子也揭示了大前提,\quotation{若 $A$ 则 $B$}表示 $B$ 仅仅是 $A$ 的逻辑结果,并非必须是因果意义上的物理结果——$B$ 晚于 $A$ 方有效。上午十点下雨并非上午 9:45 多云的物理原因。然而,但是正确的逻辑联系并非错误的因果指向(多云 $\Rightarrow$ 雨),而是(雨 $\Rightarrow$ 多云)这种非因果但正确的指向。

之所以在一开始便强调逻辑联系,是因为关于推断 \footnote{译注:注意推断与推理的区别} 的一些讨论与应用陷入了严重的误区。这些误区正是由于未能辨清逻辑蕴含与物理因果之间的区别而造成的。Simon 与 Rescher (1996) 深入分析了二者的区别。他们注意到,像表达物理因果那样解释逻辑蕴含的所有尝试,都会因为缺乏第二种三段论 (\in[syllogism-2]) 这种逆否形式而失败。也就是说,如果我们尝试将大前提解释为\quotation{$A$ 是 $B$ 的物理原因},那么就很难接受\quotation{非 $B$ 是非 $A$ 的物理原因}。在第 3 章中可以看到,以物理因果的形式来解释可信推断,也没有取得更好的进展。

另一个弱三段论,其大前提未变,

\placeformula[weak-2]
\startformula
\syllogism{若 $A$ 为真,则 $B$ 为假}{$A$ 为假}{所以,$B$ 更不可信}
\stopformula

那位警察的推理并非来自上述的弱三段论,而是来自更弱的三段论:

\placeformula[weak-3]
\startformula
\syllogism{若 $A$ 为真,则 $B$ 更可信}{$B$ 为真}{所以,$A$ 更可信}
\stopformula

且不论这一论证外在的弱点,依循它,警察的结论就会有着很强的说服力。有某种东西让我们相信,在特定情况下,这种论证有着几乎与演绎推理同样有力。

这些例子展现了,在可信推理中,大脑不仅要判断事物变得更可信还是更不可信,还要以某种方式估算可信程度。上午十点下雨可信程度要依赖于上午 9:45 的云层的黑暗程度。并且,大脑会像利用这个问题的具体的新数据那样利用旧的信息;我们尽力回忆过去与多云和雨有关的经验来决定该做什么,也依据昨晚的天气预报来决定该做什么。

要阐明警察的确要用到他们过去所累积的经验,必须改变一下他们的经验。假设像那样的事件每天晚上都被每位警察碰到——绅士每次都能证明自己是完全无罪的。很快,警察就会主动忽略这样的琐事。

因此,我们的推理极为仰仗先验信息,借助它完成新问题中的可信程度估计。这种推理过程常于不知不觉甚至瞬间完成;我们将其称为常识。这一称谓掩盖了它的复杂性。

数学家 George Pólya 为可信推理写了三卷书,给出了丰富有趣的例子,并表明可信推理是基于确定的规则进行的(尽管在他的著作中,它们表现为定性的形式)。上述的弱三段论均出现于第三卷书。我们强烈建议读者能查阅 Pólya 的著作,它是本书中许多观点的源头。在后文中,我们会表明 Pólya 的规则是可以量化的,并且这种量化规则非常有用。

显然,上文所述的演绎推理具有这样的性质,通过 (\in[syllogism-1]) 与 (\in[syllogism-2]) 所形成的推理链,可以得出像前提条件那样确定的结论。若使用其他类别的推理,诸如上述的三种弱三段论,结论的可靠性会在不同的推理阶段而发生变化。但是若使用 (\in[weak-1])--(\in[weak-3]) 这些弱三段论的量化形式,在许多情况下,我们的结论就能够达到演绎推理的确定程度(就像警察的那个例子让我们所期待的那样)。Pólya 指出,甚至一个研究纯数学的数学家实际上大部分时间也是在使用这些更弱的推理方式。当然,在公布一个新定理时,数学家将会努力发明一个论证,使之服从强三段论;但是初步导出定理的推理过程几乎总是包含着更弱的推理形式(例如,对类比所引发的猜想的探究)。S. Banach (被 S. Ulam 引用,1957) 的评论表达了类似的观点:好的数学家看到的是定理之间的相似之处;伟大的数学家看到的是相似之处之间的相似之处。

\section{类比于物理学}

在物理学中,很快就会发现世界超级复杂,不可能一下分析清楚,除非是将它剖分为一些小片,逐一研究。惟有如此,方能取得进展。有时会发明一种数学模型,用它去重现这些小片的某些特征,这时就会觉得取得了一些进展。这类模型被称为物理理论。随着知识的推进,所发明的模型越来越好,它们重现了现实中越来越多的特征,并且越来越精确。没人知道这个过程会不会自然终止,或无休止地进行下去。

在试图理解常识的过程中,我们会采用类似的思路。我们不打算一口吃成胖子,但是一旦能够构造出理想的能够重现有关常识的部分特征的数学模型,就会觉得有所进步。我们期盼这样的模型在未来能够被更完美的模型取代,也知道这一探索过程会不会自然终止。

我们的工作与物理学的相似之处不仅仅是探索方法。我们所熟悉的东西经常让我们觉得非常难以理解。那些不为常人所知的现象(诸如铁与镍的紫外光谱的不同)反而能够通过数学形式精确的描述。然而,所有现代科学,在遇到如何描述一枚草叶的生长这种寻常的现象时就变成英雄无用武之地了。因此,我们不会对这样的模型寄予太多厚望,必须做好心理准备,在构造任何一种像样的模型时,心智活动的最为寻常的特征都有可能会为最大的困难。

还有更多的相似之处。在物理学中,令我们习以为常的是,知识所取得的任何一点进步都会产生巨大但不可预料的实用价值。伦琴发现了 X 光,令医疗诊断重要革新成为可能。麦克思韦所发现的旋度方程的第二项\footnote{译注:应该是位移电流。},令瞬时通信呼之欲出。

我们为常识建立的数学模型也具备这种实用特征。任何成功的模型,哪怕它只能重现常识的少量特征,也是某一应用领域中常识的强大扩展。在该领域内,它让我们能够求解一些推断方面的问题。若没有它的帮助,对于这些被复杂细节卷裹的问题,解决起来就会一筹莫展。

\section{会思考的计算机}

许多人热衷于宣称,永远也不可能造出替代人脑的机器——人脑能干的许多事,机器永远也做不到。冯·诺依曼于 1948 年在普林斯顿所作的一次报告中对这一言论给出了巧妙的回应,我有幸聆听了那场报告。作为对来自听众的代表性问题(那个问题是\quotation{不过是一台机器,它不可能真正思考,对吧?})的回答,他说:你坚持有的事是机器做不到的。倘若你能把机器做不到的那件事精确地告诉我,我就总能造出专门来做那件事的机器。

原则上,一台机器无法完成的运算正是那些我们无法详尽描述的运算,或不能在有限步骤下完成的运算。可能有人会提到歌德尔不完备性,不可判定性,以及永不停机的图灵机等。要回答这类质疑,只需指出这些东西是因人脑的存在而生。正如冯·诺依曼所言,要制造会思考的机器,最根本的限制在于我们尚未确切理解\quotation{思维}的构成。

然而在我们对常识的研究中,我们将会触摸到一些与思维机制有关的非常清晰的观点。每次通过预定义一组明确的运算,将一个能重现部分常识的数学模型建立起来,它就会向我们展示如何\quotation{构造一台机器}(例如,写一个计算机程序)。这台机器在不完备的信息之上运转,并通过那几条弱三段论的量化版本进行可信推理,而不是演绎推理。

实际上,对于某些特定的推断问题,像这样的计算机软件的开发是这个领域中最为活跃且有用的趋势之一。所处理的一类问题可能是,给定大量数据,其中包含 10 000 独立的观测,依据这些数据以及手头上掌握的任何先验信息,确定与数据起因相关的 100 种不同的可行假设的可信程度。

对于两种泾渭分明的假设的判断,个人的常识可能就足够了;但是面对 100 种假设,它们之间的界限也不是那么清晰,若没有计算机,也没有一种发达的数学理论来指导编程,那么我们就会不知所措。也就是说,在那位警察的弱三段论中,究竟是什么决定了 A 的可信度大幅提升,甚至提升到了必然的程度;或者只是提升了微不足道的一点点,让 B 近乎与 A 无关呢?这本书的目标就是在当前可能的深度与广度上建立能够回答上述问题的数学理论。

尽管我们期盼得到一种对计算机编程有用的数学理论,但是会思考的计算机的这一观点也有助于在心理学方面建立一种数学理论。人脑所用的推理过程,这方面的问题受情感和谬论的主宰。讨论这方面的问题,很难不卷入到一些问题的争论之中。不过这些问题,就我们所掌握的知识而言还是不足以探讨清楚,而且也与我们的目的无关。

显然,人脑中的运算相当复杂,以至于我们无需假装去解释它的神秘之处,并且在任何情况下,也不会尝试去解释人类大脑里所有难以重现的迷乱和不一致性。这虽然是个有趣并且重要的主题,但是它并非我们在此所研究的主题。我们的主题是逻辑的规范性原理,并非心理学或神经生理学原理。

为了强调这一点,我们要问的并非是,\quotation{如何构建人类常识的数学模型},而是问\quotation{如何构建一台机器,让它在清晰定义可表示理想化的常识的原理下,去执行有用的可信推理}。

\section{介绍一下机器人}

为了搁置争议,专心致志,我们要发明一种虚构的东西。它的大脑由我们来设计,这样它就可以根据某些确定的规则进行推理了。这些规则可从一些简单的基本假设推导出来。在我们看来,这些基本假设是人心所向。亦即,我们认为一个理性的人,一旦意识到他的思维违背了这些基本假设,就会想对自己的思维进行修正。

原则上,我们可以采用我们喜欢的任何规则。这些规则的定义,也就是我们将要研究的那个机器人的定义。将我们的机器人所作的推理与你所作的推理进行比较,如果你发现二者并无相似之处,那么你有理由拒绝我们的机器人,然后你可以根据你的想法再设计一个新的机器人。不过,如果你发现二者非常相似,那么就会信任这个机器人,需要让它帮助你去解决你所面对的那些推断问题。总之,理论终将获得成功,而不仅仅停留在假设阶段。

我们的机器人作的是关于命题的推理。目前我们必须要求,对于这个机器人而言,所涉及的任何命题必须有着明确的含义,必须是简单的、确定的、非真即假的逻辑形式。亦即,除非特殊声明,我们只关心二值逻辑,或亚里士多德式的逻辑。像这样的一个\quotation{亚里士多德命题},是正确的还是错误的,我们不关心。实际上,我们对此无能无力,正是我们要所发明这个机器人的主要原因。例如,我个人认为下面两个命题皆为真:


\startformula
\startalign
\NC A \NC ≡ \text{\bf 贝多芬与柏辽兹素未谋面。}\NR
\NC B \NC ≡ \text{\bf 尽管柏辽兹在其全胜时期不逊色于任何人,但贝多芬的音乐较柏辽兹更具有存世价值。}\NR
\stopalign
\stopformula

命题 $B$,目前并非我们的机器人所能推理的命题,而命题 $A$ 却是,尽管要判定该命题的真伪看上去有些不太可能。在我们的理论建成之后,你会发现,有趣的是,像 $A$ 这样的亚里士多德命题,其严格性可以放宽;如此一来,机器人便可以处理像 $B$ 这样的命题了。

\section{布尔代数}

为了更正式地阐述这些想法,需要引入通用的符号逻辑或布尔代数的记法,之所如此称呼,是因为乔治·布尔用过一套相似的记法。当然,演绎逻辑自身的原理在布尔之前便已广为人知,并且马上就会看到,布尔代数的运算规则不过是拉普拉斯(1812)给出的可信推断的规则的特例罢了。符号

\placeformula
\startformula
AB
\stopformula

称为{\bf 逻辑乘}或{\bf 合取},所表示的命题是\quotation{$A$ 与 $B$ 皆真}。显然,把它们的次序颠倒一下也没关系;$AB$ 与 $BA$ 是一回事。下面这个表达式

\placeformula
\startformula
A + B
\stopformula

被称为{\bf 逻辑和}或{\bf 析取},表示\quotation{$A$ 与 $B$ 至少有一个为真},$B + A$ 同理。这些符号仅仅是为了便于命题的书写,它们并不表示数值。

给定的两个命题 $A$ 与 $B$,有一种情况是,当且仅当其中一个为真时,另一个为真;此时,我们称二者具有相同的{\bf 真值}。这种情况可能仅仅是一种简单的重言(例如,$A$ 与 $B$ 在字面上说的是一回事),也可能是经过了繁琐的数学推导最终证明 $A$ 是 $B$ 的充分必要条件。从逻辑学的观点来看,无论用什么方法,只要确定 $A$ 与 $B$ 具有相同的真值,那么它们就是逻辑等价的命题。任何可证明一方为真的证据也能证明另一方为真。在更进一步的推理过程中,$A$ 与 $B$ 具有相同的含义。

显然,具有相同真值的两个命题,它们是同样可信的。可信推理必须将此作为最基本的公理。若非布尔本人(1854, p. 286)出过错,这原本不值一提。布尔错误地将两个原本不同的命题视为等价,结果发现它们的可信度有所不同,但不知矛盾之所在。三年之后,布尔用修正的理论取代了早期著作中的理论;有关此事的评述,见 Keynes(1921, pp. 167-168),Jaynes(1976,pp. 240-242)。

在布尔代数中,等号所表示的并非数值相等,而是真值相等。像 $A = B$ 这样的布尔代数方程,表示左侧命题与右侧命题具有相同的真值。符号\quotation{$\equiv$}表示\quotation{定义相同}。

在表示复杂的命题时,我们沿用普通代数中的括号用法来描述命题组合的顺序(有时无需括号但仅仅出于清晰表述的需要而用)。若没有括号,那么代数层次规则就变得像便携计算器里那样:$AB + C$ 表示的是 $(AB) + C$,而不是 $A(B + C)$。

短横线表示对命题的否定:

\placeformula[1-8]
\startformula
\itbar{A} \equiv A\text{ 为假}
\stopformula

$A$ 与 $\itbar{A}$ 成相反关系:

\placeformula
\startformula
A = \itbar{A}\text{ 为假}
\stopformula

不管短横线落在方程的哪一侧,都不会影响这个关系的成立。注意,对于短横线的使用要谨慎一些。例如,基于上述约定,有

\placeformula
\startformula
\itbar{AB} = AB\text{ 为假}
\stopformula

\placeformula
\startformula
\itbar{A}\itbar{B} = A\text{ 与 }B\text{ 皆为假}
\stopformula

$\itbar{AB}$ 与 $\itbar{A}\itbar{B}$ 是不同的命题,前者并非 $\itbar{A}$ 与 $\itbar{B}$ 的逻辑积,而是它们的逻辑和,即 $\itbar{AB} = \itbar{A} + \itbar{B}$。

基于上述知识,布尔代数可借助一些相当简明的基本等式来体现,这些等式表示以下性质:

\placeformula[1-12]
\startformula
\startalign[n=2,align={left,left}]
\NC\text{幂等性:}
\NC\quad\startmathcases
    \NC AA = A\NR
    \NC A + A = A\NR
\stopmathcases\NR

\NC\text{交换性:}
\NC\quad\startmathcases
    \NC AB = BA\NR
    \NC A + B = B + A\NR
\stopmathcases\NR
				   
\NC\text{结合性:}
\NC\quad\startmathcases
    \NC A(BC) = (AB)C = ABC\NR
    \NC A + (B + C) = (A + B) + C = A + B + C\NR
\stopmathcases\NR

\NC\text{分配性:}
\NC\quad\startmathcases
    \NC A(B + C) = AB + AC\NR
    \NC A + (BC) = (A + B)(A + C)\NR
\stopmathcases\NR

\NC\text{对偶性:}
\NC\quad\startmathcases
    \NC\text{若 } C = AB\text{,则 } \itbar{C} = \itbar{A} + \itbar{B}\NR
    \NC \text{若 } D = A + B\text{,则 } \itbar{D} = \itbar{A}\itbar{B}\NR
\stopmathcases\NR
\stopalign
\stopformula

基于这些性质,可以证明无限多的更复杂的关系。例如,我们将要用到的一个很基本的定理:

\placeformula[1-13]
\startformula
\text{若 }\itbar{B} = AD\text{,则 }A\itbar{B} = \itbar{B}\text{ 且 }B\itbar{A} = \itbar{A}
\stopformula

\blank[big]
\middlealigned{\bf 蕴含}
\blank[small]

下面这个命题

\placeformula[implication]
\startformula
A\Rightarrow B
\stopformula

可读作\quotation{$A$ 蕴含(Imply) $B$},它并非断言 $A$ 为真或 $B$ 为真,仅仅意味着 $A\itbar{B}$ 为假或 $(\itbar{A} + B)$ 为真。这个命题也可以写为逻辑方程 $A = AB$。也就是说,对于 (\in[implication])而言,若 $A$ 为真,则 $B$ 必为真;或者,若 $B$ 为假,则 $A$ 必为假。这恰好对应强三段论 (\in[syllogism-1]) 与 (\in[syllogism-2])。

从另一个角度看,若 $A$ 为假,则 (\in[implication]) 与 $B$ 无关;若 $B$ 为真,上述命题与 $A$ 无关。这恰好对应着我们之前提出的弱三段论 (\in[weak-1]) 与 (\in[weak-2])。从某种意义上说,\quotation{弱三段论}这个术语容易令人产生误解。基于弱三段论的可信推理理论并非逻辑学的\quotation{弱化}形式,而是逻辑学的扩展;它所关心的内容,传统演绎逻辑无法涵盖。在下一章就会揭示(见 (\in[2-69]) 和 (\in[2-70]))演绎逻辑不过是可信推理的一种特例。

\blank[big]
\middlealigned{\bf 棘手之处}
\blank[small]

要注意,在日常语言中,有人会将\quotation{蕴含}理解为\quotation{暗示}或\quotation{意味},从而将 $A\Rightarrow B$ 理解为在逻辑上 $B$ 可由 $A$ 推导出来。然而在正式的逻辑学中,\quotation{$A$ 蕴含 $B$}仅意味着命题 $A$ 与 $AB$ 有相同的真值。一般而言,$B$ 在逻辑上是否可由 $A$ 推导出来,只有命题 $A$ 与 $B$ 是不够的,它依赖我们信以为真并且可在推导过程所需的全部命题 $(A, A', A'',\cdots)$。Devinatz(1968, p. 3)与 Hamilton(1988, p. 5)将逻辑蕴含视为一种二元运算,给出了它的真值表;结果表明 $A\Rightarrow B$ 只有在 $A$ 为真且 $B$ 为假时为假,在其他情况下 $A\Rightarrow B$  皆为真!

乍看上去令人惊讶;然而,实际上,若 $A$ 与 $B$ 皆为真,则 $A = AB$,因而 $A\Rightarrow B$ 为真;在形式逻辑中,每个真实陈述都能蕴含其他的真实陈述。另一方面,若 $A$ 为假,则对于所有 $Q$ 而言,$AQ$ 为假,因而 $A = AB$ 且 $A = A\itbar{B}$ 皆为真;一个虚假命题能够蕴含所有命题。如果我们将蕴含理解为逻辑推导(亦即 $B$ 与 $\itbar{B}$ 皆由 $A$ 推导出来),结果就是每个虚假命题在逻辑上必然能够导出矛盾。然而,像\quotation{贝多芬活得比柏辽兹更久}这样的虚假命题在逻辑上却很难导出矛盾(因为贝多芬活得比许多与柏辽兹同龄的人更久)。

显然,仅仅知道命题 $A$ 与 $B$ 皆为真,该信息不足以用于判断二者之一在逻辑上是否可从另一个推导出来。从一个命题集合从逻辑上推导出一个命题,这一问题在第二章中所讨论的歌德尔定理中再作详细介绍。\quotation{蕴含}这个单词在日常语言与逻辑学中含义上的显著不同是令人头疼的问题,若不能正确的认识到这一点,便会引出严重的错误;当初选用\quotation{蕴含}这个词并不妥当,在逻辑学的传统的介绍中没有充分强调这一点。

\section{运算的完备集}

我们关心的是设计我们的机器人所需要的演绎逻辑的一些特征。我们已经定义了四种运算或\quotation{联结词},通过它们,可从命题 $A$ 和 $B$ 开始,定义其他命题:逻辑乘或合取 $AB$,逻辑和或析取 $A + B$,蕴含 $A\Rightarrow B$,以及取反 $\itbar{A}$。以所有可能的方式重复组合这些运算,便可生成任意个新命题,诸如

\placeformula[1-15]
\startformula
C\equiv (A + \itbar{B})(\itbar{A} + A\itbar{B}) + \itbar{A}B(A + B)
\stopformula

于是就出现了许多问题:这样生成的新命题集合的规模有多大?这个集合是有限集合么,或者是否存在对这些运算封闭的集合?能这样表示每个由 $A$ 和 $B$ 定义的命题么,或者要做到这一点是否需要除上述四个联结词之外的联结词?或者,这四个联结词是否已经过多因而需要有所扬弃呢?足以生成 $A$ 和 $B$ 的所有这样的\quotation{逻辑函数}的最小运算集是怎样的?如果初始命题并非 $A$ 和 $B$,而是任意个数的 $\{A_1,\cdots,A_n\}$,那么这个运算集是否依然能够生成 $\{A_1,\cdots,A_n\}$ 的所有可能的逻辑函数?

从逻辑学、概率论以及计算机设计的角度来看,这些问题并不难以回答。就我们现在所知而言,这些问题大体可归结为 (1) 能否增加函数的数量,(2) 能否减少运算的数量。采用 (\in[1-15]) 的方式构造的两个命题,若它们有相同的真值,那么它们就是相同的命题,基于这一点便可简化第一个问题。例如 (\in[1-15]) 的 $C$ 所描述的东西在逻辑上与蕴含 $C = (B\Rightarrow \itbar{A})$ 相同,这一点留待读者验证。

由于现阶段我们的关心的只是亚里士多德命题,任何像 (\in[1-15]) 这样的逻辑函数 $C = f(A, B)$ 仅有两个可能的\quotation{值},真与假;同样地,\quotation{独立变量}$A$ 和 $B$ 的值也只能从这二者中去取。

至此,逻辑学家可能会反对我们的记法,称符号 $A$ 已经用于指代为某个确定的命题了,其存在性已不可变;因此,不应当将一个逻辑函数写为 $C = f(A, B)$,而应当引入新的符号,将其写为 $z = f(x, y)$,其中 $x$,$y$,$z$ 是\quotation{陈述变量},用于替换 $A$,$B$,$C$ 这些特定的陈述。然而,若 $A$ 表示某个确定但非特定的命题,它依然可以是真,也可以是假。我们只需要将像 (\in[1-15]) 这样的方程理解为对于定义 $A$ 和 $B$ 的所有方式皆为真,便可以达到相同的效果;亦即,我们用的不是陈述变量,而是可变的陈述。

在形式 $C = f(A, B)$ 的关系中,我们关心定义于离散\quotation{空间} $S$ 之上的逻辑函数,该离散空间由 $2^2 = 4$ 个点构成,亦即这些逻辑函数定义在 $A$ 和 $B$ 取值为 $\{TT, TF, FT, FF\}$ 时所构成的空间上;并且在每个点上,$f(A, B)$ 可独立地从 $\{T,F\}$ 中取值。因而,刚好有 $2^4 = 16$ 个逻辑函数\footnote{译注:4 个二进制位的编码空间。},不会更多。形如 $B = f(A_1, \cdots, A_n)$ 这种囊括 $n$ 个命题的表示,是定义在由 $2^M$ 个点构成的离散空间上的逻辑函数,并且恰好有 $2^M$ 个这样的函数。

在 $n = 1$ 的情况中,有四个逻辑函数 $\{f_1(A),\cdots,f_4(A)\}$,它们可能的取值形成下面的真值表:

%%%%%% 三线表示例
%\setupxtable[offset=4pt,align={broad,middle,lohi},frame=off]
%\placetable[none][tab:wealthdecline]
%    {}
%    {\startxtable
%      \startxtablehead[foregroundstyle=bold,topframe=on, rulethickness=1.5pt]
%      \startxrow
%          \startxcell[nx=2] \ConTeXt\ 文稿各级标题宏 \stopxcell
%          \startxcell[ny=2,width=2.5cm] 含义 \stopxcell
%      \stopxrow
%      \startxrow[foregroundstyle=bold, rulethickness=0.75pt]
%          \startxcell 有编号标题 \stopxcell
%          \startxcell 无编号标题 \stopxcell
%      \stopxrow
%      \stopxtablehead
%      \startxtablebody[align={broad,ragleft,lohi}]
%      \startxrow[topframe=on,rulethickness=0.75pt]
%          \startxcell \type{\part} \stopxcell
%          \startxcell  \stopxcell
%          \startxcell[topframe=on,rulethickness=0.75pt] 部分 \stopxcell
%      \stopxrow
%      \stopxtablebody
%      \startxtablefoot[bottomframe=on,rulethickness=1.5pt]
%      \startxrow
%          \startxcell \type{\subsubsubsubsection} \stopxcell
%          \startxcell \type{\subsubsubsubsubject} \stopxcell
%          \startxcell 次次次小节 \stopxcell
%      \stopxrow
%      \stopxtablefoot
%      \stopxtable}

\placetable[none][tab:wealthdecline]{}
{\startxtable
  \startxtablehead[topframe=on, rulethickness=1.5pt]
  \startxrow
  \startxcell[width=2cm] $A$ \stopxcell
  \startxcell[width=2cm] T \stopxcell
  \startxcell[width=2cm] F \stopxcell
  \stopxrow
  \stopxtablehead
  \startxtablebody
  \startxrow[topframe=on,rulethickness=0.75pt]
  \startxcell $f_1(A)$ \stopxcell
  \startxcell T  \stopxcell
  \startxcell T \stopxcell
  \stopxrow
  \startxrow
  \startxcell $f_2(A)$ \stopxcell
  \startxcell T  \stopxcell
  \startxcell F \stopxcell
  \stopxrow
  \startxrow
  \startxcell $f_3(A)$ \stopxcell
  \startxcell F \stopxcell
  \startxcell T \stopxcell
  \stopxrow          
  \stopxtablebody
  \startxtablefoot[bottomframe=on,rulethickness=1.5pt]
  \startxrow
  \startxcell $f_4(A)$ \stopxcell
  \startxcell F \stopxcell
  \startxcell F \stopxcell
  \stopxrow
  \stopxtablefoot
  \stopxtable
}

根据真值表,可通过穷举的方法定义这 4 个逻辑函数。通过检验,显然它们恰好是

\placeformula
\startformula
\startalign
\NC f_1(A) \NC = A + \itbar{A}\NR
\NC f_2(A) \NC = A\NR
\NC f_3(A) \NC = \itbar{A}\NR
\NC f_4(A) \NC = A\itbar{A}\NR
\stopalign
\stopformula

因此,我们穷举证实了通过三种运算——合取、析取、取反,足以生成单个命题的全部逻辑函数。

对于一般性的 $n$ 而言,首先考虑一些特定函数,它们仅在 $S$ 的一个点上为真。当 $n = 2$ 时,有 $2^n = 4$ 个这样的函数,其真值表如下:

\placetable[none][tab:wealthdecline]{}
{\startxtable
  \startxtablehead[topframe=on, rulethickness=1.5pt]
  \startxrow
  \startxcell[width=2cm] $A$, $B$ \stopxcell
  \startxcell[width=2cm] TT \stopxcell
  \startxcell[width=2cm] TF \stopxcell
  \startxcell[width=2cm] FT \stopxcell
  \startxcell[width=2cm] FF \stopxcell
  \stopxrow
  \stopxtablehead
  \startxtablebody
  \startxrow[topframe=on,rulethickness=0.75pt]
  \startxcell $f_1(A)$ \stopxcell
  \startxcell T \stopxcell
  \startxcell F \stopxcell
  \startxcell F \stopxcell
  \startxcell F \stopxcell
  \stopxrow
  \startxrow
  \startxcell $f_2(A)$ \stopxcell
  \startxcell F \stopxcell
  \startxcell T \stopxcell
  \startxcell F \stopxcell
  \startxcell F \stopxcell
  \stopxrow
  \startxrow
  \startxcell $f_3(A)$ \stopxcell
  \startxcell F \stopxcell
  \startxcell F \stopxcell
  \startxcell T \stopxcell
  \startxcell F \stopxcell
  \stopxrow          
  \stopxtablebody
  \startxtablefoot[bottomframe=on,rulethickness=1.5pt]
  \startxrow
  \startxcell $f_4(A)$ \stopxcell
  \startxcell F \stopxcell
  \startxcell F \stopxcell
  \startxcell F \stopxcell
  \startxcell T \stopxcell
  \stopxrow
  \stopxtablefoot
  \stopxtable
}

通过检验,它们恰好是 4 个基本的合取运算:

\placeformula[1-17]
\startformula
\startalign
\NC f_1(A) \NC = AB\NR
\NC f_2(A) \NC = A\itbar{B}\NR
\NC f_3(A) \NC = \itbar{A}B\NR
\NC f_4(A) \NC = \itbar{A}\itbar{B}\NR
\stopalign
\stopformula

现在考虑在 $S$ 上某些特定的点上的任一逻辑函数;例如,$f_5(A, B)$ 和 $f_6(A, B)$:

\placetable[none][tab:wealthdecline]{}
{\startxtable
  \startxtablehead[topframe=on, rulethickness=1.5pt]
  \startxrow
  \startxcell[width=2cm] $A$, $B$ \stopxcell
  \startxcell[width=2cm] TT \stopxcell
  \startxcell[width=2cm] TF \stopxcell
  \startxcell[width=2cm] FT \stopxcell
  \startxcell[width=2cm] FF \stopxcell
  \stopxrow
  \stopxtablehead
  \startxtablebody
  \startxrow[topframe=on,rulethickness=0.75pt]
  \startxcell $f_5(A)$ \stopxcell
  \startxcell F \stopxcell
  \startxcell T \stopxcell
  \startxcell F \stopxcell
  \startxcell T \stopxcell
  \stopxrow
  \stopxtablebody
  \startxtablefoot[bottomframe=on,rulethickness=1.5pt]
  \startxrow
  \startxcell $f_6(A)$ \stopxcell
  \startxcell T \stopxcell
  \startxcell F \stopxcell
  \startxcell T \stopxcell
  \startxcell T \stopxcell
  \stopxrow
  \stopxtablefoot
  \stopxtable
}

我们断言,每一个这样的函数都是 (\in[1-17]) 的这些合取运算的逻辑和,它们在相同的点上为真(这并非无足轻重,读者应当验证一番)。因而,

\placeformula
\startformula
\startalign
\NC f_5(A, B) \NC = f_2(A, B) + f_4(A, B)\NR
\NC \NC = A\itbar{B} + \itbar{A}\itbar{B}\NR
\NC \NC = (A + \itbar{A})\itbar{B}\NR
\NC \NC = \itbar{B}\NR
\stopalign
\stopformula

同样地,

\placeformula
\startformula
\startalign
\NC f_6(A, B) \NC = f_1(A, B) + f_3(A, B) + f_4(A, B)\NR
\NC \NC = AB + \itbar{A}B + \itbar{A}\itbar{B}\NR
\NC \NC = B + \itbar{A}\itbar{B}\NR
\NC \NC = \itbar{A} + B\NR
\stopalign
\stopformula

更确切地说,$f_6(A, B)$ 是 $f_6(A, B) = (A\Rightarrow B)$,即蕴含,其值如上述真值表所示。在 $S$ 的至少一点上为真的任一逻辑函数 $f(A, B)$ 皆可用 (\in[1-17]) 这些基本合取的逻辑和的方式来构造。像这样的函数共有 $2^4 - 1 = 15$ 个。剩下的那个函数,它的值总是为假,足以用来表示矛盾,$f_{16}(A, B) = A\itbar{A}$。

这种方法(在逻辑学教科书里将其称为\quotation{约化至析取范式})适用于任意 $n$。例如,在 $n = 5$ 的情况中,有 $2^5 = 32$ 个基本的合取

\placeformula
\startformula
\{ABCDE,\,ABCD\itbar{E},\,ABC\itbar{D}E,\cdots,\itbar{A}\itbar{B}\itbar{C}\itbar{D}\itbar{E}\}
\stopformula

还有 $2^{32} = 4\,294\,967\,296$ 个不同的逻辑函数 $f_i(A,B,CD,E) = A\itbar{A}$,其中有 $4\,294\,967\,295$ 个可以写成基本合取的逻辑和,只剩下矛盾:

\placeformula
\startformula
f_{4294967296} = A\itbar{A}
\stopformula

因而,通过\quotation{思维中的建构}便可验证三种运算

\placeformula[1-23]
\startformula
\{\text{合取,析取,取反}\},\quad\text{亦即}\quad \{\text{与,或,非}\}
\stopformula

足以生成所有可能的逻辑函数;或者,简单地说,它们形成了一个运算的完备集。

(\in[1-12]) 的对偶性揭示了一个更小的集合的存在,因为 $A$ 和 $B$ 的析取同于对它们皆为假的否定:

\placeformula
\startformula
A + B = \overline{(\itbar{A}\itbar{B})}
\stopformula

因而,两种运算 $(\text{与,非})$ 就足以构成演绎逻辑的完备集\footnote{需要思考:这是否意味着,写任何一个计算机程序,只需要两条命令?}。在确定可信推理规则的完备集时,这一事实非常重要;见第 \in[quantitative-rules] 章。

显然,现在的这些运算,我们无法剔除一个,而只保留另一个,亦即,\quotation{与}运算不能约化为取反,而取反也不能通过任意数量的\quotation{与}运算来表示。但是,这并没有否定合取与取反这两种运算可以约化为第三种尚未引入的运算的可能性,以致单一的逻辑运算便可构成完备集。

令人惊喜的是,这样的运算不止一个,而是两个。\quotation{与}的取反,可称为\quotation{与非}运算:

\placeformula
\startformula
A\uparrow B = \itbar{AB} = \itbar{A} + \itbar{B}
\stopformula

我们将其读为\quotation{$A$ 与非 $B$}。不过,这样就有

\placeformula
\startformula
\startalign
\NC \itbar{A} \NC = A \uparrow A\NR
\NC \itbar{AB} \NC = (A\uparrow B) \uparrow (A\uparrow B)\NR
\NC \itbar{A + B} \NC = (A\uparrow B) \uparrow (B\uparrow B)\NR
\stopalign
\stopformula

因而,只用与非便可以构造出全部的逻辑函数。同样地,或非运算

\placeformula
\startformula
A\downarrow B = \itbar{A + B} = \itbar{A}\itbar{B}
\stopformula

也强大到了足以生成所有逻辑函数的程度:

\placeformula
\startformula
\startalign
\NC \itbar{A} \NC = A \downarrow A\NR
\NC \itbar{A + B} \NC = (A\downarrow B) \downarrow (A\downarrow B)\NR
\NC \itbar{AB} \NC = (A\downarrow B) \uparrow (B\downarrow B)\NR
\stopalign
\stopformula

在设计计算机或逻辑电路时,可以利用这一优势。\quotation{逻辑门}是一个电路,除了接地线之外,有两个输入终端和一个输出终端。这些终端的任意一个,其对地电压只呈现两个值:$+3$ 伏,或称上升沿,表示\quotation{真};$0$ 伏,或称下降沿,表示\quotation{假}。一个与非门,当且仅当它的一个输入处于下降沿,它的输出是上升沿;这种效果等效的是,当且仅当它的两个输入皆为上升沿,它的输出是下降沿,而对于或非门而言,当且仅当两个输入皆为下降沿,其输出是上升沿。

逻辑电路的一个标准组件是\quotation{四与非门},即一个半导体芯片上的集成电路含有 4 个独立的与非门。这样的元件,若数量足够多,只需将它们用不同的方式相互连接起来,就可以生成任何所需的逻辑函数,无须借助其他电路组件。

关于演绎逻辑的这一趟短途旅程,对于我们的目的而言已经足够了。更深入的知识有许多教科书可以参考;例如 Copi (1994) 给出了亚里士多德逻辑的现代处理方式。对于非亚里士多德形式,并在哥德尔不完备性、可计算性、可确定性、图灵机等问题上有特殊强调的问题,可参考 Hamilton (1988)。

现在,我们将注意力转向我们对逻辑所作的扩展。这一扩展遵循下面要讨论的一些条件。我们将这些条件称为\quotation{祈求}而非\quotation{公理},因为它们并不断言何事为真,而是仅陈述怎样才是令人满意的目标。这些目标是否能够无矛盾地达成,是否确定任何唯一的逻辑扩展,是数学分析上的事情,而这些分析在第 \in[quantitative-rules] 章给出。

\section{基本祈求}

我们的机器人必须基于我们给它的证据将可信度赋予所推理的每个命题,并且无论何时,它接受了新的证据,就必须对原先为命题所赋的可信度进行更改,以契合新的证据。为了所赋的可信性能够在机器人大脑电路里存储和修改,它们必须与一些明确的物理量存在联系,诸如电压、脉冲持续时间或二进制编码的数字等——不过,这些事应该交给工程师。就当前目的而言,我们只需要可信度与实数之间必须存在某种联系:

\placeformula[desiderata-1]
\startformula
\text{(\Roman{1})}\quad\text{\it 可信度用实数表示。}
\stopformula

实际上,祈求 (\Roman{1}) 是强加在我们身上的,因为机器人的大脑必须通过执行一些明确的物理过程方能运转。不过,这也是理论上的需要(见附录 A);任何具有一致性的理论,不可能不具备与祈求 (\Roman{1}) 等价的性质。

我们采纳了一个自然但并非必须的约定:可信性越强,对应的数越大。假设存在连续性也会方便一些,只是眼下尚无法阐述清楚这一点;直观地说:比无限小大一点的可信性应该只对应于比无限小大一点的数。

机器人赋予某个命题 $A$ 的可信性,通常依赖于我们是否曾告诉它的某个命题 $B$ 为真。沿用 Keynes (1921) 与 Cox (1961) 的记法,我们用符号将此事表示为

\placeformula
\startformula
A|B
\stopformula

我们将此式称为\quotation{给定 $B$ 为真,$A$ 为真的条件可信性},或简称为\quotation{$A$ 受于 $B$}\footnote{译注:取自古代统治者宣称君权\quotation{受命于天}的说法。}。它代表着某个实数。因而,例如

\placeformula
\startformula
A|BC
\stopformula

(可将其读为\quotation{$A$ 受于 $BC$})表示在给定 $B$ 与 $C$ 为真的情况下 $A$ 为真的可信性。或者,

\placeformula
\startformula
A + B|CD
\stopformula

表示在给定 $C$ 与 $D$ 为真的情况下 $A$ 与 $B$ 至少有一个为真的可信性;凡此种种,不一而足。由于我们已经决定用更大的数来表示更大的可信性,因此

\placeformula[1-32]
\startformula
(A|B) > (C|B)
\stopformula

说的是,给定 $B$,$A$ 比 $C$ 更可信。在这种记法中,尽管不带括号的 $A|B$ 这种形式就是在表示可信性,但是为了表达上的清晰,我们经常会给它加上括号。因而,(\in[1-32]) 等同于

\placeformula
\startformula
A|B > C|B
\stopformula

不过,(\in[1-32]) 看起来会更清晰一些。

为了避开不可能的问题,我们不打算让这个机器人陷入源自不可能或互相矛盾的前提的推理噩梦;这些噩梦不可能有\quotation{正确}答案。因此,如果 $B$ 和 $C$ 相互矛盾,我们就不去定义 $A|BC$。无论何时,当这样的符号出现的时候,就意味着 $B$ 和 $C$ 是相容的命题。

还有,我们不希望这个机器人的思考方式与你和我的思考方式南辕北辙。因此,我们会把它设计成以人类的推理方式去推理,至少在定性方面如此,如前文所述的弱三段论与一些类似方式。

因而,如果它有旧的信息 $C$,这个信息以提升 $A$ 的可信性的方式被更新为 $C'$:

\placeformula
\startformula
(A|C') > (A|C)
\stopformula

但是,$B$ 受于 $A$ 的可信性并未变化:

\placeformula
\startformula
(B|AC') = (B|AC)
\stopformula

这可以让\quotation{$A$ 与 $B$ 皆为真}的可信性只会提升,不会下降:

\placeformula
\startformula
(AB|C') \ge (AB|C)
\stopformula

\quotation{$A$ 为假}的可信性一定会下降:

\placeformula
\startformula
(\itbar{A}|C') < (\itbar{A}|C)
\stopformula

这种定性需求简单地给出了\quotation{方向感},使得机器人的推理能够进行下去。上式并未给出可信性的变化过程。前面所作的连续性假设(这也是定性符合常识的条件之一)要求:在 $A|C$ 的可信性微小变化的情况下,机器人仅让 $AB|C$ 与 $\itbar{A}|C$ 发生微小变化。在下一章,我们会展示如何使用这些定性要求,也会说明为什么需要它们。此刻,我们将这些要求简单概括为:

\placeformula[desiderata-2]
\startformula
\text{(\Roman{2})}\quad\text{\it 定性符合常识。}
\stopformula

最后,我们想让这个机器人再多一个值得拥有的属性:总是能一致地作推理。这个属性,真诚的人,心之所向,不能常至。我们对\quotation{一致性}的理解表述为以下三个方面:


\startsubformulas[desiderata-3a]
\placeformula
\startformula
\text{(\Roman{3}a)}\quad\quad\hbox to .75\hsize{\startmathmatrix[n=2, align={left}]
\NC \text{\it 如果一个结论能够通过多种方式推理出来,那么每一种可能方式}\NR
\NC \text{\it 的结果必定相同。}\NR
\stopmathmatrix}
\stopformula

\placeformula[desiderata-3b]
\startformula
\text{(\Roman{3}b)}\quad\quad\hbox to .75\hsize{\startmathmatrix[n=2, align={left}]
\NC \text{\it 这个机器人总是会考虑所有与问题相关的证据。它不会随意忽略}\NR
\NC \text{\it 某些信息,仅基于剩下的信息给出结论。换言之,它是完全非意}\NR
\NC \text{\it 识形态化的}\NR
\stopmathmatrix}
\stopformula

\placeformula[desiderata-3c]
\startformula
\text{(\Roman{3}c)}\quad\quad\hbox to .75\hsize{\startmathmatrix[n=3, align={left}]
\NC \text{\it 这个机器人总是通过相等的可行性赋值来表示相等的知识状态。}\NR
\NC \text{\it 亦即,若它对两个问题所具备的知识状态相同(除了命题的标记}\NR
\NC \text{\it 有所区别之外),它就必须为这两个问题赋以相同的可信性。}\NR
\stopmathmatrix}
\stopformula
\stopsubformulas

祈求 (\Roman{1})、(\Roman{2}) 和 (\Roman{3}a) 是基本的\quotation{结构}要求,为我们的机器人大脑的内部运作而设,而 (\Roman{3}b) 和 (\Roman{3}c) 是\quotation{界面}条件,揭示了机器人的行为应当如何与外界建立联系。

大多数学生会惊讶于我们对祈求的探寻至此而终。上述条件,唯一性地确定了我们的机器人在作推理时凭借的规则;亦即,操控可信性的数学运算集仅有一个。这些规则在第 \in[quantitative-rules] 章中推出。

(许多章的末尾,我们会给出一些非正式的评论,其中包含了我们所搜集的一些不同角度的谈论、背景材料等。忽略这些并不影响本书的主线。)

\section{评论}

出于维护各种各样的政治、经济、道德、宗教、心理、环境、饮食以及艺术的教条化的地位,政治家、广告商、推销员和传教士深谙易犯错的人类的思想易于愚弄。他们凭借聪明的措辞,令人忘乎所以从而悖离上述祈求。我们要试图保证我们的机器人不会轻易让他们得逞。

我们要强调一下机器人与人脑的另一个区别。根据祈求 (\Roman{1}),机器人对于任何命题的心智状态皆表示为实数。显然,对于任何给定的命题,我们的看法有多个\quotation{坐标}。对于一个命题,你和我(同时)形成的判断,不仅关乎它是否可信,也关乎它是否值得,是否重要,是否有用,是否有趣,是否好笑,是否道德正确,等等。如果我们假设每一个这样的判断可以表示为一个数字,那么就可以将人类思想状态的完备描述表示为一个维数相当大的空间中的向量。

并非所有命题需要这样。例如,命题\quotation{水的折射率小于 1.3}不会让人产生情感;由它产生的思想状态的坐标的数量会很少。相反,\quotation{你的岳母弄坏了你的新车}就会生成带有许多坐标的思想状态。日常生活中的事往往牵涉许多坐标。正是出于这一原因,我们认为,让模型最为难以重现的往往是心智活动的那些最为寻常的例子。科学与数学之所以是人类最为成功的活动,原因在于:它们处理命题,这些命题产生的心智状态最为简单。这种状态不会被给定的人类思想缺陷轻易动摇。

当然,许多目的决定了我们不想让这个机器人采纳这些来自其他坐标的更为\quotation{人性}的特征。计算机不痴,不嗔,不贪,这些特质使得它们在执行特定任务时比人类更为可靠。

插入这些话,旨在指出对于本书要开发这个理论而言,存在一个巨大的有待探索的领域——我们的理论的泛化和扩展。或许这样会鼓舞他人着手发展心智活动的\quotation{多维理论},使之更为贴近真实人脑的行为——它们并非完全地不受欢迎。像这样的理论,一旦成功,其重要性超乎我们的想象。\footnote{实际上,一些心理学家认为只需要五个维度便足以概括人的性格;也就是说,我们的差异仅仅在于五种性格基本特质混合上的不同,这些性格特质由基因决定。但是,在我们看来,这一结论必定过度简化了事实;可辨识的化学因素在时间和空间里连续变化(诸如大脑中的葡萄糖代谢的分布)影响着心智活动,但是很难相信仅用五维空间便能表示这一切。不过,对于许多目的而言,五个数字所捕捉的真相可能也足够了。}

不过,目前我们现在只能做一些力所能及的工作。开发具有一致性的\quotation{一维}可信推理模型,这个想法行得通吗?显然,如果能设法唯一地将可信度表示为单个实数,并忽略刚才提到的其他\quotation{坐标},那么我们的问题就会相当简单。

我们强调的是,我们还没有办法断定真正的人类思想中的可信度有着唯一的数值度量。我们的工作不是去假设——抑或推测——任何像这样的事。我们的工作是去{\bf 调查}一下,能否在机器人内部建立这样的不矛盾的对应关系。

但是,可能在某些人看来,我们已经假设了太多东西,因而在我们的理论的通用性方面施加了不必要的限制。为什么必须可信性要用实数表示?基于诸如 $(A|C) > (B|C)$ 这样的量化排序系统的\quotation{比较}理论还不够用么?这一点在附录 A 中作了进一步讨论,其中我们描述了通过其他途径建立概率论,并给出了致力于建构被认为逻辑上更为简单且更为通用的比较理论的一些尝试。但是,实际情况不尽人意;因此,尽管很有可能通过其他方式形成比我们的假设更好的基础,但终究不会有差别。

\subsection{通用语言 vs 形式逻辑}

我们应当注意形式逻辑的陈述与日常语言的陈述之间的区别。可能有人会认为后者不过是一种不精确的表达而已;但是经过细致的检查,似乎并不简单。在我们看来,日常语言,若谨慎使用,未必比形式逻辑逊色;只是日常语言的规则更复杂且比形式逻辑更富有表现力。

实际上,通用语言(译注:普通话),应用总是比逻辑学更为广泛,其精微玄妙——隐喻或暗指——是形式逻辑所缺乏的。A 先生,为了表示自信,说\quotation{我相信我看到的}。B 先生反驳:\quotation{他看不到他不相信的}。从形式逻辑的角度来看,他们俩说的是一回事;但是从通用语言的角度来看,这些话有着反义的意图和效果。

一本数学教科书里有一个不一般的例子。令 $L$ 是平面上的一条直线,$S$ 是平面上的无限点集,将 $S$ 中的每个点投影到 $L$ 上。来看下面的两句话:

\startitemize[R]
\item 极限的投影是投影的极限。
\item 投影的极限是极限的投影。
\stopitemize

它们的语法结构为\quotation{$A$ 是 $B$}和\quotation{$B$ 是 $A$},因此它们可能表现为逻辑意义上的相等。然而在那本教科书里,(\Roman{1}) 是成立的,(\Roman{2}) 通常不成立,因为 $S$ 可以没有极限,但它在 $L$ 上的投影可能是有极限。

如我们所见,在通用语言中——甚至在数学教科书里——我们已经学会了理解这些精微玄妙时加入确切的语义,可能是无意识的,除非给出一个像这样的例子指明这一点。我们对\quotation{$A$ 是 $B$}解释,首先是断言 $A$ 的存在,然后以此为大前提,剩下的句子被理解为取决于这个前提。换言之,在通用语言中,动词\quotation{是(Is)}意味着主语与宾语的区别,形式逻辑或传统数学中所用的符号\quotation{=}却没有这种用法。(不过,在计算机语言中,我们遇到过诸如\quotation{\type{J = J + 1}}这样的语句,似乎都理解这样的语句,但是其中的符号\quotation{=}已经隐含了这种区别。)

另一个可笑的例子是过去的格言\quotation{知识就是力量},不论是在人际关系里还是在热力学里,这都是中肯的大实话。一个广告写手为一家化工贸易杂志把这句话糟蹋成\quotation{权力就是知识},好一个荒谬——甚至下流——的篡改。

这些例子提醒了我们,动词\quotation{是},像其他动词一样,有一个主语和一个谓语;但是很少有人注意到这个动词有着两种完全不同的意思。母语是英语的人也许需要一些努力才能在语句中察觉这两种不同的意思:\quotation{The room is noisy(房间太吵)}和\quotation{There is noise in the room(房间内有噪音)}。但是在土耳其语里,这两种意思由不同的词来表达,这样就可以让区别相当明显,若是有人用错了词,他的话会令人无法理解。后面那句话是本体论意义上的描述,断言某种事物的物理意义上的存在,而前面那句话是认识论意义上的描述,表达的仅仅是说话的人的个人感受。

通用语言——或者,至少,英语——有一个近乎普遍的倾向,对认识论意义上的陈述进行伪装,方法是把它置入合乎语法的形式,冒充本体论意义的描述,从而具备了欺骗性。当前的概率论中,错误的主要根源在于欠缺思考而忽略了这一事实。用本体论的方式来解释第一种陈述,如同宣称一个人私下的想法和感觉是自然界中外在存在的事实。我们将这种现象称为\quotation{思想投射谬误},并注意到它如何再三制造麻烦。但是这种麻烦并不仅仅局限于概率论;这种现象一经指出,哲学和格式塔心理学方面的大多数讨论,以及物理学家量子力学的解释,将这一切简单地解释为作者反复陷入思想投射谬误就是显而易见的了。

要求我们的机器人能够理解通用语言的所有精微玄妙之处,这就太过分了,毕竟人类需要用 20 多年才能达到这种程度。在这方面,我们的机器人会一直像一个小孩——它从字面解读所有语句,并且心直口快,毫不避讳。

这些例子说明了当我们尝试将通用语言的复杂语句翻译为形式逻辑的更为简化的语句事务必谨慎。当然,通用语言的准确性往往低于比我们在形式逻辑中所期望的水平。但是每个人意识并且留心这一点,风险就会很小。

对于本书作者而言,不清楚,甚至也不觉得有必要去设计一种更新的模型机器人,令它更为善解人意。当然,这在原理上不成问题,毕竟人脑的存在证实了这一点。但是,冯·诺依曼的原理符合实际;我们设计的机器人做不到这一点,除非有人提出一种\quotation{玄妙识别}理论,将这个过程简化为运算的确定描述集。我们乐于将此事让于他人。

无论如何,我们当前的模型机器人都极为现实,因为任何重要的概率估计交由计算机来完成在如今已近乎普及。给计算机编程的人,无论他是否这样认为,都是在根据所预想的机器人应当如何作为来设计一个机器人大脑的一部分。但是,现在正在投入使用的计算机程序,只有极少数满足我们全部的祈求;实际上,大部分都是直觉意义上的{\bf 应景}过程,选择它们,根本不是出于任何良好定义的祈求。

任何这样的应景方案可能在特殊的应用领域有用——这是选择应景方案的标准——但是,第 \in[quanlititive-rules] 章的证明将会表明,任何与概率论规则冲突的应景方案,当我们越界使用它的时候,一定会造成明显的不一致性。我们的目标是直接从一致性要求出发,并且以适用于任何可信推断问题的形式——以足够清楚的方式公式化,一劳永逸地提出通用的推断原理。

\subsection{吹毛求疵}

如前文所述,在这本书里,我们使用术语\quotation{布尔代数}是因袭其公认的含义,即二值逻辑,将命题表似乎为\quotation{$A$}这样的符号。一个过于挑剔的人说,一些数学家以一种稍微不同的含义使用这个术语,\quotation{$A$}这样的符号用来指代一类命题。不过,这两种用法并不冲突;我们认可这种更为广泛的含义,只是并未发现它对我们有何益处。

我们称为\quotation{布尔代数}的这套规则和符号有时也叫\quotation{命题演算}。要使用这个术语,我们还需要其他的规则和符号,它们被称为\quotation{谓词演算}。然而,这些新的符号不过是一些我们所熟悉的短语的简称罢了。\quotation{全称量词}仅仅是\quotation{对于所有的}的简称;\quotation{存在量词}是\quotation{存在一个}的简称。如果只是用平实的英语来写东西,就已经自动在使用我们所需要的全部谓词演算了,并且更为清晰。

第二个强三段论(在二值逻辑里)的有效性经常受到质疑。然而,通过反例去证明一个假设的定理是错的,认为可推导出矛盾的陈述集合是不一致的,通过归缪法——从反命题推导出矛盾的方式确定去一个命题为真,这些在数学里依然被认为是有效的推理。对我们而言,这就足够了;我们非常满足于遵循这一悠久的传统。我们在这一立场中的安全感来自于这样的信仰,逻辑是前进的而不是后退的。新的逻辑可能会到处亚里士多德逻辑无法言说的结果;事实上,这也正是我们在此努力去创造的东西。但是,确切地说,如果新的逻辑与亚里士多德逻辑在某个领域有发生冲突,而亚里士多德逻辑适用于这个领域,那么我们就会认为新的逻辑必定有着致命的硬伤。

因而,我们只能对那些觉得受二值演绎逻辑所束缚的人说:\quotation{如果你愿意,大可以去研究其他的可能性;一旦你有了二值逻辑或者我们的扩展所没有的新发现,并且这一发现对科学推断有用,还望不吝赐教。}实际上,现有的文献里已经提出了许多不同的并且相互不一致的多值逻辑。但是在附录 A 中,我们援引了一些论据,这些论据认为多值逻辑比二值逻辑多出来的那些东西没有什么用处;也就是说,$n$ 值逻辑适用于一个命题集,这等价于二值逻辑适用于一个更大的命题集,否则,这个 $n$ 值逻辑的内部存在不一致性。

我们的经验与这样的猜测一致:在实践中,多值逻辑似乎并未用于发现新的有用的结果,而是尝试消除那些假设在二值逻辑中存在的困难,在量子力学、模糊集合以及人工智能等里此举尤甚。但是,随着研究的深入,这些困难,据我们所知,不过是思想投射谬误的一些例子,只需要概念方面的修正,并不需要一门新的逻辑学。