\chapter[plausible-reasoning]{可信推理}

\startdictum
逻辑学所关心的问题,要么是肯定或否定的,要么是完全不确定的。总之,(应当庆幸)没什么问题需要推理。世上真正的逻辑应该是概率计算——关乎可能性的量化运算,它存在甚至应该存在于理性者的思维之中。
  
\rightaligned{James Clerk Maxwell,1850}
\stopdictum

在漆黑的夜晚,一名警察在空无一人街道上巡逻。突然,他听到了警铃惊响,循声望去,街对面一家珠宝店的窗户洞开,一位蒙面的绅士,肩负一袋昂贵的珠宝,向外爬出。警察不假思索,立即断定此人绝非善类。他是基于什么得出这一结论的呢?

\section{演绎推理与可信推理}

警察的结论显然不是通过从证据出发的逻辑推导而得到。任何犯罪,皆能为之编造出完美的无罪辩护。譬如,那位绅士可能是珠宝店的主人。当天夜里,他从一个化妆舞会回家,发现没带钥匙。当他走到自己的店铺时,一辆驶过的卡车扔出一块石头击破了窗户。他为了保护自己的财产不得已而上演了警察所看到的一幕。

尽管警察的推理过程并非逻辑推导,但是我们得承认他的结论在一定程度上也是正确的。警察看到的那一幕,不能作为证据,去证明那位绅士犯罪,但是却能够让自己的结论看上去更为可信。这种推理方式或许是我们的本能。有些事,例如今天是否下雨,由于缺乏足够的信息,无法进行演绎推理,但是又必须当机立断。这样的事情在我们的清醒的时间里几乎无时无刻不在发生。

这种推理貌似寻常,其过程却非常微妙。尽管相关的讨论已经延续了 24 个世纪,但迄今依然没有令人满意的答案。在这本书里,我们会汇报一些有用且令人鼓舞的新的进展,即使用明确的定理去代替那些不靠谱的直觉判断,并且使用一些非常基本且近乎无法避免的理性标准所确定的规则代替那些特设的过程。

有关这些问题的所有讨论,始于几个例子,它们反映了演绎推理与可信推理的对立。演绎推理(证明式推理),亦即亚里士多德(公元前 4 世纪)\footnote{现在,对于亚里士多德的贡献的确切性有一些不同的看法。这方面的争议与我们开展的工作无关,但是感兴趣的读者可以从 Lukasiewicz(1957)那里找到这方面的一些讨论。} 的工具论,总是能够归结为两个强三段论的反复运用,即

\placeformula[syllogism-1]
\startformula
\syllogism{若 $A$ 为真,则 $B$ 为真}{$A$ 为真}{所以,$B$ 为真}
\stopformula

与

\placeformula[syllogism-2]
\startformula
\syllogism{若 $A$ 为真,则 $B$ 为真}{$B$ 为假}{所以,$A$ 为假}
\stopformula

这就是我们乐意时时使用的推理方式,但是很不幸,大部分情况下我们缺乏这种推理所需要的信息,只好退而求其次,使用弱三段论:

\placeformula[weak-1]
\startformula
\syllogism{若 $A$ 为真,则 $B$ 为真}{$B$ 为真}{所以,$A$ 更可信}
\stopformula

证据不足以证明 $A$ 为真,但是若 $A$ 的结果之一得到确证,我们就会对 $A$ 为真更有信心。例如,设

\startformula
\startalign
\NC A\equiv\NC\text{\bf 最晚上午十点下雨。}\NR
\NC B\equiv\NC\text{\bf 上午十点之前多云。}\NR
\stopalign
\stopformula

在上午 9:45 看到云,并不能断定将会下雨。我们的常识是服从弱三段论的,它会督促我们改变计划与行动,因此如果那些云足够黑,我们就会相信会下雨。

这个例子也揭示了大前提,「若 $A$ 则 $B$」表示 $B$ 仅仅是 $A$ 的逻辑结果,并非必须是因果意义上的物理结果——$B$ 晚于 $A$ 方有效。上午十点下雨并非上午 9:45 多云的物理原因。然而,但是正确的逻辑联系并非错误的因果指向(多云 $\Rightarrow$ 雨),而是(雨 $\Rightarrow$ 多云)这种非因果但正确的指向。

之所以在一开始便强调逻辑联系,是因为关于推断 \footnote{译注:注意推断与推理的区别} 的一些讨论与应用陷入了严重的误区。这些误区正是由于未能辨清逻辑蕴含与物理因果之间的区别而造成的。Simon 与 Rescher (1996) 深入分析了二者的区别。他们注意到,像表达物理因果那样解释逻辑蕴含的所有尝试,都会因为缺乏第二种三段论 (\in[syllogism-2]) 这种逆否形式而失败。也就是说,如果我们尝试将大前提解释为「$A$ 是 $B$ 的物理原因」,那么就很难接受「非 $B$ 是非 $A$ 的物理原因」。在第 3 章中可以看到,以物理因果的形式来解释可信推断,也没有取得更好的进展。

另一个弱三段论,其大前提未变,

\placeformula[weak-2]
\startformula
\syllogism{若 $A$ 为真,则 $B$ 为假}{$A$ 为假}{所以,$B$ 更不可信}
\stopformula

那位警察的推理并非来自上述的弱三段论,而是来自更弱的三段论:

\placeformula[weak-3]
\startformula
\syllogism{若 $A$ 为真,则 $B$ 更可信}{$B$ 为真}{所以,$A$ 更可信}
\stopformula

且不论这一论证外在的弱点,依循它,警察的结论就会有着很强的说服力。有某种东西让我们相信,在特定情况下,这种论证有着几乎与演绎推理同样有力。

这些例子展现了,在可信推理中,大脑不仅要判断事物变得更可信还是更不可信,还要以某种方式估算可信程度。上午十点下雨可信程度要依赖于上午 9:45 的云层的黑暗程度。并且,大脑会像利用这个问题的具体的新数据那样利用旧的信息;我们尽力回忆过去与多云和雨有关的经验来决定该做什么,也依据昨晚的天气预报来决定该做什么。

要阐明警察的确要用到他们过去所累积的经验,必须改变一下他们的经验。假设像那样的事件每天晚上都被每位警察碰到——绅士每次都能证明自己是完全无罪的。很快,警察就会主动忽略这样的琐事。

因此,我们的推理极为仰仗先验信息,借助它完成新问题中的可信程度估计。这种推理过程常于不知不觉甚至瞬间完成;我们将其称为常识。这一称谓掩盖了它的复杂性。

数学家 George Pólya 为可信推理写了三卷书,给出了丰富有趣的例子,并表明可信推理是基于确定的规则进行的(尽管在他的著作中,它们表现为定性的形式)。上述的弱三段论均出现于第三卷书。我们强烈建议读者能查阅 Pólya 的著作,它是本书中许多观点的源头。在后文中,我们会表明 Pólya 的规则是可以量化的,并且这种量化规则非常有用。

显然,上文所述的演绎推理具有这样的性质,通过 (\in[syllogism-1]) 与 (\in[syllogism-2]) 所形成的推理链,可以得出像前提条件那样确定的结论。若使用其他类别的推理,诸如上述的三种弱三段论,结论的可靠性会在不同的推理阶段而发生变化。但是若使用 (\in[weak-1])--(\in[weak-3]) 这些弱三段论的量化形式,在许多情况下,我们的结论就能够达到演绎推理的确定程度(就像警察的那个例子让我们所期待的那样)。Pólya 指出,甚至一个研究纯数学的数学家实际上大部分时间也是在使用这些更弱的推理方式。当然,在公布一个新定理时,数学家将会努力发明一个论证,使之服从强三段论;但是初步导出定理的推理过程几乎总是包含着更弱的推理形式(例如,对类比所引发的猜想的探究)。S. Banach (被 S. Ulam 引用,1957) 的评论表达了类似的观点:好的数学家看到的是定理之间的相似之处;伟大的数学家看到的是相似之处之间的相似之处。

\section{类比于物理学}

在物理学中,很快就会发现世界超级复杂,不可能一下分析清楚,除非是将它剖分为一些小片,逐一研究。惟有如此,我们才能取得进展。有时会发明一种数学模型,它能够重现这些小片的某些特征,此时就会觉得取得了一些进展。此类模型被称为物理理论。随着知识的推进,所发明的模型越来越好,它们重现了现实中越来越多的特征,并且越来越精确。没人知道这个过程会不会自然终止,或无休止地进行下去。

在试图理解常识的过程中,我们会采用类似的思路。我们不打算一口吃成胖子,但是一旦能够构造出理想的能够重现有关常识的部分特征的数学模型,我们就会感到有所进步。我们期盼这样的模型在未来能够被更完美的模型取代,也知道这一探索过程会不会自然终止。

我们的工作与物理学的相似之处不仅仅是探索方法。经常性地,我们所熟悉的东西让我们感到非常难以理解。那些不为常人所知的现象(诸如铁与镍的紫外光谱的不同)能够通过数学形式精确的描述,然而所有现代科学,在遇到如何描述一枚草叶的生长这种寻常的现象时,则全然无用。因此,我们不会对这个模型寄予过多厚望,必须做好心理准备,在构造任何一种像样的模型时,心智活动的最为寻常的特征都有可能会为最大的困难。

还有更多的相似之处。在物理学中,令我们习以为常的是,知识所取得的任何一点进步都会产生巨大但不可预料的实用价值。伦琴发现了 X 光,令医疗诊断重要革新成为可能。麦克思韦所发现的旋度方程的第二项(译者注:应该是位移电流),令瞬时通信呼之欲出。

我们为常识建立的数学模型也具备这种实用特征。任何成功的模型,哪怕它只能重现常识的少量特征,也会有力扩展某一应用领域中的常识。在该领域内,它让我们能够求解一些推断方面的问题。若没有它的帮助,对于这些被复杂细节卷裹的问题,我们没有任何头绪去解决。

\section{会思考的计算机}

许多人热衷于宣称,永远也不可能造出替代人脑的机器——人脑能干的许多事,机器永远也做不到。冯·诺依曼于 1948 年在普林斯顿所作的一次报告中对这一言论给出了巧妙的回应,我有幸聆听了那场报告。作为对来自听众的代表性问题(那个问题是「不过是一台机器,它不可能真正思考,对吧?」)的回答,他说:你坚持有的事是一台机器做不到的。如果你能将一台机器做不到的那件事精确地告诉我,那么我总能制造一台专门来做那件事的机器。

原则上,一台机器无法完成的运算正是那些我们无法详尽描述的运算,或不能在有限步骤下完成的运算。可能有人会提到歌德尔不完备性,不可判定性,以及永不停机的图灵机等。要回答这类质疑,只需指出这些东西是因人脑的存在而生。正如冯·诺依曼所言,要制造会思考的机器,最根本的限制在于我们未能确切的理解「思维」的构成。

然而在我们对常识的研究中,我们将会触摸到一些与思维机制有关的非常清晰的观点。每次通过预定义一组明确的运算,将一个能重现部分常识的数学模型建立起来,它就会向我们展示如何「构造一台机器」(例如,写一个计算机程序)。这台机器在不完备的信息之上运转,并通过那几条弱三段论的量化版本进行可信推理,而不是演绎推理。

实际上,对于某些特定的推断问题,像这样的计算机软件的开发是这个领域中最为活跃且有用的趋势之一。所处理的一类问题可能是,给定大量数据,其中包含 10 000 独立的观测,依据这些数据以及手头上掌握的任何先验信息,确定与数据起因相关的 100 种不同的可行假设的可信程度。

对于两种泾渭分明的假设的判断,个人的常识可能就足够了;但是面对 100 种假设,它们之间的界限也不是那么清晰,若没有计算机,也没有一种发达的数学理论来指导编程,那么我们就会不知所措。也就是说,在那位警察的弱三段论中,究竟是什么决定了 A 的可信度大幅提升,甚至提升到了必然的程度;或者只是提升了微不足道的一点点,让 B 近乎与 A 无关呢?这本书的目标就是在当前可能的深度与广度上建立能够回答上述问题的数学理论。

尽管我们期盼得到一种对计算机编程有用的数学理论,但是会思考的计算机的这一观点也有助于在心理学方面建立一种数学理论。人脑所用的推理过程,这方面的问题受情感和谬论的主宰。讨论这方面的问题,很难不卷入到一些问题的争论之中。不过这些问题,就我们所掌握的知识而言还是不足以探讨清楚,而且也与我们的目的无关。

显然,人脑中的运算相当复杂,以至于我们无需假装去解释它的神秘之处,并且在任何情况下,也不会尝试去解释人类大脑里所有难以重现的迷乱和不一致性。这虽然是个有趣并且重要的主题,但是它并非我们在此所研究的主题。我们的主题是逻辑的规范性原理,并非心理学或神经生理学原理。

为了强调这一点,我们要问的并非是,「如何构建人类常识的数学模型」,而是问「如何构建一台机器,让它在清晰定义可表示理想化的常识的原理下,去执行有用的可信推理」。

\section{介绍一下机器人}

为了搁置争议,专心致志,我们要发明一种虚构的东西。它的大脑由我们来设计,这样它就可以根据某些确定的规则进行推理了。这些规则可从一些简单的基本假设推导出来。在我们看来,这些基本假设是人心所向。亦即,我们认为一个理性的人,一旦意识到他的思维违背了这些基本假设,就会想对自己的思维进行修正。

原则上,我们可以采用我们喜欢的任何规则。这些规则的定义,也就是我们将要研究的那个机器人的定义。将我们的机器人所作的推理与你所作的推理进行比较,如果你发现二者并无相似之处,那么你有理由拒绝我们的机器人,然后你可以根据你的想法再设计一个新的机器人。不过,如果你发现二者非常相似,那么就会信任这个机器人,需要让它帮助你去解决你所面对的那些推断问题。总之,理论终将获得成功,而不仅仅停留在假设阶段。

我们的机器人作的是关于命题的推理。目前我们必须要求,对于这个机器人而言,所涉及的任何命题必须有着明确的含义,必须是简单的、确定的、非真即假的逻辑形式。亦即,除非特殊声明,我们只关心二值逻辑,或亚里士多德式的逻辑。像这样的一个「亚里士多德命题」,是正确的还是错误的,我们不关心。实际上,我们对此无能无力,正是我们要所发明这个机器人的主要原因。例如,我个人认为下面两个命题皆为真:


\startformula
\startalign
\NC A \NC ≡ \text{\bf 贝多芬与柏辽兹素未谋面。}\NR
\NC B \NC ≡ \text{\bf 尽管柏辽兹在其全胜时期不逊色于任何人,但贝多芬的音乐较柏辽兹更具有存世价值。}\NR
\stopalign
\stopformula

命题 $B$,目前并非我们的机器人所能推理的命题,而命题 $A$ 却是,尽管要判定该命题的真伪看上去有些不太可能。在我们的理论建成之后,你会发现,有趣的是,像 $A$ 这样的亚里士多德命题,其严格性可以放宽;如此一来,机器人便可以处理像 $B$ 这样的命题了。

\section{布尔代数}

为了更正式地阐述这些想法,需要引入通用的符号逻辑或布尔代数的记法,之所如此称呼,是因为乔治·布尔用过一套相似的记法。当然,演绎逻辑自身的原理在布尔之前便已广为人知,并且马上就会看到,布尔代数的运算规则不过是拉普拉斯(1812)给出的可信推断的规则的特例罢了。符号

\placeformula
\startformula
AB
\stopformula

称为{\bf 逻辑乘}或{\bf 交},所表示的命题是「$A$ 与 $B$ 皆真」。显然,把它们的次序颠倒一下也没关系;$AB$ 与 $BA$ 是一回事。下面这个表达式

\placeformula
\startformula
A + B
\stopformula

被称为{\bf 逻辑和}或{\bf 并},表示「$A$ 与 $B$ 至少有一个为真」,$B + A$ 同理。这些符号仅仅是为了便于命题的书写,它们并不表示数值。

给定的两个命题 $A$ 与 $B$,有一种情况是,当且仅当其中一个为真时,另一个为真;此时,我们称二者具有相同的{\bf 真值}。这种情况可能仅仅是一种简单的重言(例如,$A$ 与 $B$ 在字面上说的是一回事),也可能是经过了繁琐的数学推导最终证明 $A$ 是 $B$ 的充分必要条件。从逻辑学的观点来看,无论用什么方法,只要确定 $A$ 与 $B$ 具有相同的真值,那么它们就是逻辑等价的命题。任何可证明一方为真的证据也能证明另一方为真。在更进一步的推理过程中,$A$ 与 $B$ 具有相同的含义。

显然,具有相同真值的两个命题,它们是同样可信的。可信推理必须将此作为最基本的公理。若非布尔本人(1854, p. 286)出过错,这原本不值一提。布尔错误地将两个原本不同的命题视为等价,结果发现它们的可信度有所不同,但不知矛盾之所在。三年之后,布尔用修正的理论取代了早期的书的理论;有关此事的评述,见 Keynes(1921, pp. 167-168),Jaynes(1976,pp. 240-242)。

在布尔代数中,等号所表示的并非数值相等,而是真值相等。像 $A = B$ 这样的布尔代数方程,表示左侧命题与右侧命题具有相同的真值。符号「$\equiv$」表示「定义相同」。

在表示复杂的命题时,我们沿用普通代数中的括号用法来描述命题组合的顺序(有时无需括号但仅仅出于清晰表述的需要而用)。若没有括号,那么代数层次规则就变得像便携计算器里那样:$AB + C$ 表示的是 $(AB) + C$,而不是 $A(B + C)$。



\section{运算的完备集}

\section{公设}

\section{评论}