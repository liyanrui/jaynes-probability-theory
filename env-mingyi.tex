\startenvironment env-mingyi
\usemodule[zhfonts]

\setupzhfonts
  [serif]
  [bold=kaiti,
   italic=kaiti,
   bolditalic=kaiti]
%\setupzhfonts[math][modem]
\setupmathematics[integral=nolimits] % 调整积分的上下界位置

\setuppapersize[A4][A4]
\definelayout
  [cover]
  [backspace=0.05\paperwidth, width=0.95\paperwidth]
\setuplayout[backspace=26mm, 
             width=158mm, 
             topspace=30mm,
             hight=242mm,
             header=15mm, 
             footer=10mm]
\setuppagenumbering[alternative=doublesided]

% 页眉页脚
\setuppagenumbering[location=]

\newdimen\headerwidth
\headerwidth=\the\makeupwidth
\newdimen\footerwidth
\footerwidth=\the\makeupwidth
\advance\footerwidth by .9\rightmarginwidth
\newdimen\LineWidth
\LineWidth=1pt

\startluacode
function zhvert(arg)
    for c in arg:utfcharacters() do
        if #c >=3 then
            context.rotate({rotation='90'}, c)
            context([[\kern.2em]])
        else
            context([[\kern.5em ]])
        end
    end
end
\stopluacode
\def\zhvert#1{\ctxlua{zhvert('#1')}}

\def\HeaderFrame#1{\framed[width={\headerwidth}, frame=off, offset=none]{\bf#1}}
\def\PageNumberFrame{\inframed[frame=off, offset=0pt]{\tfx\pagenumber}}
\def\HeadStr#1{\headnumber[#1]\hskip1em\getmarking[#1]}
\def\RightHeader{\HeaderFrame{\HeadStr{section}\hfill\PageNumberFrame\hbox to -\LineWidth{}}}
\def\LeftHeader{\HeaderFrame{\hbox to -\LineWidth{}{\PageNumberFrame}\hfill\HeadStr{chapter}}}

\def\FooterFrame#1{%
  \framed[width={\footerwidth}, frame=off, offset=0pt]{#1}}
\def\BookNameFrame[#1]{%
  \framed[width=fit, height=fit, frame=off, offset=0pt]{%
    \hskip2em\rotate[rotation=#1]{\zhvert{概率论}\raise.15em\hbox{——}\zhvert{科学逻辑}}\hskip2em}}
\def\RightFooter{\FooterFrame{\hfill\BookNameFrame[-90]}}
\def\LeftFooter{\FooterFrame{\BookNameFrame[-90]\hfill}}

% 标题
\setupheads[indentnext=yes]
\definepagebreak[headpagebreak][yes, header, footer, odd]
\setuphead
  [chapter,title]
  [header=empty, 
    style=\bfd, 
    page=headpagebreak,
    after={\blank[2cm]}]
\setuphead[chapter][separator=000]

\startuseMPgraphic{DancingBox}
path p;
u := \overlaywidth; v := \overlayheight;
p := fullsquare xyscaled (u, v) randomized 0.07u;
drawpath p; drawpoints p;
\stopuseMPgraphic
\defineoverlay[FrameBG][{\useMPgraphic{DancingBox}}]
\def\ContentTitle#1{%
  \framed
    [width=fit,
      height=fit,
      align=middle,
      frame=off,
      offset=4pt,
      loffset=1em,
      roffset=1em,
      background=FrameBG]{#1}}
\def\ChapterNumber#1{\hbox{\color[middlegray]{#1}}}
\define[2]\titlecmd{\ContentTitle{#2}}
\define[2]\chaptercmd{\hbox to\hsize{%
    \hfill\ContentTitle{#1\hskip1em #2}\hfill}}
\setuphead[title][command=\titlecmd]
\setuphead[chapter][command=\chaptercmd]
\setuphead[section, subject][style=\bfb]

\startsetups Text
\setupheadertexts[text][\RightHeader][][][\LeftHeader]
\setupfootertexts[text][\RightFooter][][][\LeftFooter]
\stopsetups

% 段落
\setupindenting[first,always,2em]
\setupinterlinespace[line=1.5em]

% 摘要
\definestartstop
  [dictum]
  [before={\startnarrower[4*middle]}, after={\stopnarrower\blank[big]}]

% 列表
\setupitemize[paragraph, packed, broad]
\setupitemize
  [left=(,
    right=),
    margin=4em,
    stopper=,
    distance=0em,
    itemalign=flushright]

% misc
\setupheadtext[en][pubs=参考文献]
\setupheadtext[en][content=目录]
\setupheadtext[en][index=索引]
\setuplabeltext[en][figure=图\;]
\setuplabeltext[en][table=表\;]
\setupcaptions[style=\tfx, headstyle=\normal, align={broad,middle}]

\startsetups footnote:hanzi
\setscript[hanzi]
\stopsetups
\setupnote[footnote][setups={footnote:hanzi}]

\def\itbar#1{\overline{#1\hskip 0.1em}\hskip 0.2em}

\def\syllogism#1#2#3{%
  \startmathmatrix[n=3]
    \NC\text{\framed[frame=off, offset=0pt]{#1}}\NR
    \NC\text{#2}\NR
    \NC\text{\framed[frame=off, topframe=on]{\hskip 1em #3 \hskip 1em}}\NR
  \stopmathmatrix}

\defineseparatorset[none][][]
\setupformulas[numberseparatorset=none]

% 三线表的一般设置
\setupxtable[offset=4pt,align={broad,middle,lohi},frame=off]

% 罗马数字
\def\Roman#1{\convertnumber{R}{#1}}

\stopenvironment