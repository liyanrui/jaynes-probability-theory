\startenvironment env-mingyi
\usemodule[zhfonts]

\setupzhfonts
  [serif]
  [bold=kaiti,
   italic=kaiti,
   bolditalic=kaiti]
%\setupzhfonts[math][modem]
\setupmathematics[integral=nolimits] % 调整积分的上下界位置

\setupinteraction
  [state=start,
    title={概率论},
    subtitle={科学逻辑},
    keyword={概率论, 逻辑学},
    author={E. T. Jaynes(著) 李延瑞(译)},
    focus=standard,
    style=normal]

\setupinteractionscreen[option=bookmark]
\placebookmarks[title, chapter,section][title, chapter][force=yes]

\setuppapersize[A4][A4]
\definelayout
  [cover]
  [backspace=0.05\paperwidth, width=0.95\paperwidth]
\setuplayout[backspace=26mm, 
             width=158mm, 
             topspace=18.5mm,
             header=7mm,
             headerdistance=4mm,
             height=261.5mm,
             footerdistance=2mm,
             footer=7mm]

\setuppagenumbering[alternative=doublesided]

% 页眉页脚
\setuppagenumbering[location=]
\def\innerframe#1{%
  \framed[width=broad,
    frame=off,
    height=1.3em,
    rulethickness=0.5pt,
    location=lohi,
    bottomframe=on,
    offset=overlay]{#1}}
\def\outerframe#1{%
  \framed[width=broad,
    frame=off,
    height=1.6em,
    rulethickness=0.5pt,
    location=lohi,
    bottomframe=on,
    offset=overlay]{#1}}
\def\HeaderFrame#1{%
  \outerframe{\innerframe{\switchtobodyfont[10.5pt]{#1}}}}
\def\ChapterHeaderStr{\headnumber[chapter]\hskip 1em\getmarking[chapter]}
\def\TitleHeaderStr{\getmarking[title]}
\def\BookName{\HeaderFrame{概率论——科学逻辑}}

\startsetups FrontMatter
\def\PageNumber{\switchtobodyfont[10.5pt]{\convertnumber{R}{\pagenumber}}}
\setuppagenumber[number=1]
\setupheadertexts[text][\BookName][][][\HeaderFrame{\TitleHeaderStr}]
\setupfootertexts[text][][\PageNumber][\PageNumber][]
\stopsetups

\startsetups BodyMatter
\def\PageNumber{\switchtobodyfont[10.5pt]{\pagenumber}}
\setupheadertexts[text][\BookName][][][\HeaderFrame{\ChapterHeaderStr}]
\setupfootertexts[text][][\PageNumber][\PageNumber][]
\stopsetups

% 习题
\defineenumeration
  [exercise]
  [text=, headstyle=bold, way=bychapter,
    alternative=serried, width=fit,
    headcommand={习题 \headnumber[chapter].\headnumber[exercise]}]
\definestartstop
  [Exercise]
  [before={\startframedtext[width=broad]\startexercise}, after={\stopexercise\stopframedtext}]

% 标题
\setupheads[indentnext=yes]
\definepagebreak[headpagebreak][yes, header, footer, odd]
\setuphead
  [chapter,title]
  [header=empty, 
    style=\bfd, 
    page=headpagebreak,
    after={\blank[2cm]}]

\startuseMPgraphic{DancingBox}
path p;
u := \overlaywidth; v := \overlayheight;
p := fullsquare xyscaled (u, v) randomized 0.07u;
drawpath p; drawpoints p;
\stopuseMPgraphic
\defineoverlay[FrameBG][{\useMPgraphic{DancingBox}}]
\def\ContentTitle#1{%
  \framed
    [width=fit,
      height=fit,
      align=middle,
      frame=off,
      offset=4pt,
      loffset=1em,
      roffset=1em,
      background=FrameBG]{#1}}
\def\ChapterNumber#1{\hbox{\color[middlegray]{#1}}}
\define[2]\titlecmd{\hbox to\hsize{%
    \hfill\ContentTitle{#2}\hfill}}
\define[2]\chaptercmd{\hbox to\hsize{%
    \hfill\ContentTitle{#1\hskip1em #2}\hfill}}
\setuphead[title][command=\titlecmd]
\setuphead[chapter][command=\chaptercmd]
\setuphead[section, subject][style=\bfb]

% 段落
\setupindenting[first,always,2em]
\setupinterlinespace[line=1.5em]

% 摘要
\definestartstop
  [dictum]
  [before={\startnarrower[4*middle]}, after={\stopnarrower\blank[big]}]

% 列表
\setupitemize[paragraph, packed, broad]
\setupitemize
  [left=(,
    right=),
    margin=4em,
    stopper=,
    distance=0em,
    itemalign=flushright]

% misc
\setupheadtext[en][pubs=参考文献]
\setupheadtext[en][content=目录]
\setupheadtext[en][index=索引]
\setuplabeltext[en][figure=图\;]
\setuplabeltext[en][table=表\;]
\setupcaptions[style=\tfx, headstyle=\normal, align={broad,middle}]

\startsetups footnote:hanzi
\setscript[hanzi]
\stopsetups
\setupnote[footnote][setups={footnote:hanzi}]

\def\itbar#1{\overline{#1\hskip 0.1em}\hskip 0.2em}

\def\syllogism#1#2#3{%
  \startmathmatrix
    \NC\text{\framed[frame=off, offset=0pt]{#1}}\NR
    \NC\text{#2}\NR
    \NC\text{\framed[frame=off, topframe=on]{\hbox to 1em{}#3\hbox to 1em{}}}\NR
  \stopmathmatrix}

% 插图与公式编号  x.1 -> x-1
\setupcaptions[prefixsegments=chapter,prefixconnector=-]
\setupformulas[prefixsegments=chapter,prefixconnector=-]

% 将公式符号 x.y.a -> x.ya
\defineseparatorset[none][][]
\setupformulas[numberseparatorset=none]

% 三线表的一般设置
%\setupxtable[offset=4pt,align={broad,middle,lohi},frame=off]
\setupxtable[align={broad,middle,lohi},frame=off]
\setupcaption[table][location=top]

% 罗马数字
\def\Roman#1{\convertnumber{R}{#1}}

% 组合
\definemathmatrix[pmatrix][left={\left(\,}, right={\,\right)}]

\stopenvironment