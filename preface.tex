\title{前言}

\framed[width=broad,framecolor=red,offset=1em]{目前仅翻译了第 1 节。}

本书所面向的读者,应当熟悉高年级本科及更高学位所修的应用数学,并且对于需求诸推断的某一学科也比较熟悉,诸如物理学、化学、生物学、地质学、医学、经济学、社会科学、工程学以及运筹学。不需要他们对概率与统计有太多了解,对这一学科一无所知反倒会更好一些,这样需要忘记的东西就会少一些。

我们关心的是概率论以及与之相关的传统数学,只不过与标准教科书相比,我们是要在更为广阔的场景中探讨这些数学。从第二章开始,每章都有新的东西(亦即之前的教科书未涉及的)。它们不仅有趣,而且有用。概率论的用武之地已超出了传统的概率论授课范围。但是,我们认为这些新的东西是不言自明的,与之相关的理论会成为未来的传统概率论。

\subject{历史}

本书成于多年来的日积月累。一开始,Harold Jefferys (1939) 的著作引发了我对概率论的兴趣。我觉得他的观点让理论物理中的所有问题闪现出不同于以往的光。但是很快,R. T. Cox (1949)、Shannon (1948) 以及 Pólya (1954) 等人的工作打开了新的思维空间,他们的成就已在我的脑海中盘桓了四十余载。在这个更为宏伟且永恒的广义理性思维世界里,眼下的理论物理问题仅是昙花一现。

本书伊始于 1956 年我在斯坦福大学为一系列讲座时所写的讲义,讲授当时 George  Pólya 在「数学与可信推断」方面所做的耳目一新且备受鼓舞的工作。Pólya 将我们直觉上的「常识(Common sense)」剖分为一组基本的定性要求,并表明数学家在研究发现的早期阶段一直仰仗于此,且这一阶段是在寻求严格证明之前不可或缺。这些成果颇似 James Bernolli 的 Art of Conjecture (1713) 中的工作——18 世纪 Laplace 从分析的角度对其进行了拓展;但是 Pólya 认为,二者的相似之处仅在于定性方面。

然而,Pólya 不遗余力的展现了这种定性上的契合,令人觉得其理论似具深意有待发现。所幸,R. T. Cox 的一致性定理可担此重任;将 Pólya 的定性条件加诸 R. T. Cox 的一致性定理,可证:若采用实数表示可信程度,则必定存在唯一确定的量化规则集合,该集合可用于指导推断过程。亦即,推断结果与之相悖的任何其他规则,必然会违背合理性或一致性的基本且近乎不可避免之要求。

不过,最终结果不过是概率论的标准规则,早已由 Daniel Bernoulli 与 Aaplace 给出,为何还要小题大作?其新意在于我们为推断给出了是唯一有效的广义逻辑原则,而不需要引入「机会」或「随机变量」这些概念;因此,这些逻辑原则的适用范围要比 20 世纪早期发展起来的传统概率论更为宽广。结果就是,「概率论」与「统计推断」之间的沟壑得以填平,不仅使得理论体系保持着单一性与简单性,也为其应用提供了更有力且灵活的技术支撑。

所以,我的讲座重点在于给出 Pólya 的观点的量化公式,以便将其用于一般性的科学问题的推断——将所有问题视为信息不完备所致,而非「随机性」所致。对 George Pólya 的一些追忆以及这项工作的起点,在本书的第 5 章有述。

这一理论在应用领域暂露头角之后,Harold Jeffreys 的工作便再度得到高度重视。他的直觉极为强大,我遇到的所有问题似乎他早有预料。我能做的便是写出这本书向他致以深深的敬意。有关他所做的工作的一些评述以及它们对我的影响遍布于本书多个章节。

从 1957 到 1970 这些年间,在别的一些高校以及实验室反复举办了这个讲座系列,其内容日渐丰富。随即,我们发现传统的「统计推断」中的一些难题也渐渐变得易于理解与克服了。但是,已占据了一席之地的这些规则在概念上相当巧妙,需要深入思考如何正确地运用它们。过去导致 Laplace 的工作遭遇非议的那些难题,现在看来,只不过是误用了这些规则——问题的定义不清或夸大了一些琐碎信息的作用。一旦认识到这一点,重归正途就不难了。我们的「扩展逻辑」方法与传统的「随机变量」方法之间的联系以各种形式出现于本书的每一章。

最终,集腋成裘,短期的讲座系列已不足以阐述它们,并且这项工作也从逐渐课堂中走了出来;攻克了一些难题之后,我们发现自己拥有了一套强大的工具,它们足以应对新的问题。约从 1970 年开始,内容继续增长,但其来源变为我与同事们的一些科研工作。我们希望这本书能够保持足够多的知识来源,使之既可作为教科书亦可作为参考书;实际上,已经有几届学生带走了我们早期版本的讲义,后又将其传授于他们的学生。

我们以 Charles Darwin 在介绍《物种起源》时所写的一句话作为对上文的总结:「请原谅我一开始便讲述这些个人事宜,此举旨在于表明,我给出的结论并非草率所得。」不过,由于我们的工作始于三十多年以前,在今天也许有人会认为它过时了。所幸,Jeffreys、Pólya 以及 Cox 的工作是其根基之所在,经受得起时间的考验,其真理未曾受到撼动,其重要性却与日俱增。在推断的本质上,他们的洞察,在三十多年以前曾饱受质疑,而今却在诸多科学领域大行其道;此后 100 年内,他们的工作会在所有领域发挥至关重要的作用。