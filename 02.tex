\chapter[quantitative-rules]{量化规则}

\startdictum
概率论,不过是生活约化而成的计算。
  
\rightaligned{Laplace,1819}
\stopdictum

\indentation 问题已被形式化了,这是我们所作的公设在数学上的必然结果。这些公设可粗略描述如下:

\startitemize[R]
\item 命题的可信性由实数表示;
\item 定性符合常识;
\item 一致性。
\stopitemize

\noindent 本章仅基于这些公设推导与推断相关的量化规则。所得结果,曾经有着一段漫长、复杂又令人难以置信的历史。这段历史充满着广义科学方法论方面的教训(见某些章尾部的评注部分)。

\section{乘法规则}

我们首先为逻辑乘 $AB$ 的可信性分别与 $A$ 与 $B$ 的可信性之间的关系找出一个一致性的规则。我们在意的其实是 $AB|C$。由于推理过程本身有些微妙,这需要我们从一些不同的视角去审视。

先考虑将 $AB$ 为真的裁决打碎为对 $A$ 与 $B$ 所作的更为基本的裁决。机器人可以

\startitemize[n]
\item 裁决 $B$ 为真;\hfill $(B|C)$
\item 认可 $B$ 为真,裁决 $A$ 为真。\hfill $(A|BC)$
\stopitemize

\noindent 或者,等同地,

\startitemize[n]
\item 裁决 $A$ 为真;\hfill $(A|C)$
\item 认可 $A$ 为真,裁决 $B$ 为真。\hfill $(B|AC)$
\stopitemize

\noindent 上面,在每种情况中,我们给出了每一步骤相应的可信性。

现在,解释一下第 1 个过程。为了命题 $AB$ 为真,$B$ 必须为真。因而,必须考虑 $B|C$ 的可信性。此外,若 $B$ 为真,则需要进一步保证 $A$ 为真,因此还需要考虑 $A|BC$ 的可信性。但是若 $B$ 为假,则无需裁决 $A$,亦即无需裁决 $A|\itbar{B}C$,便可知晓 $AB$ 为假;若机器人先推理 $B$,那么,$A$ 的可信性仅在 $B$ 为真时才值得考虑。因此,机器人在有了 $B|C$ 与 $A|BC$ 的情况下,就不需要 $A|C$ 了,因为 $A|C$ 并未给它能带来更多的信息。

类似地,$A|B$ 与 $B|A$ 是不需要的;无论 $A$ 与 $B$ 在缺乏 $C$ 的情况下多么可信,都与机器人在知道 $C$ 为真的情况下所作的裁决无关。例如,若机器人知晓地球是圆的,那么在裁决与现在的宇宙学相关的问题时,它就可以忽略那些在它尚不知地球是圆的之时需要考虑的观点。

由于逻辑乘法运算遵循交换律,$AB = BA$,因此上文中的 $A$ 与 $B$ 毫无疑问,可以互换;亦即 $A|C$ 与 $B|AC$ 的知识也能用于确定 $AB|C = BA|C$。对于 $AB$,机器人必定能从这两个过程中得到相同的可信性,这是我们的一致性条件之一,即公设 (\convertnumber{R}{3}a)。

可以用更明确的形式来描述。$(AB|C)$ 可以是 $B|C$ 与 $A|BC$ 的某个函数:

\placeformula[basic-function]
\startformula
(AB|C) = F[(B|C), (A|BC)]
\stopformula

现在,如果上述推理尚不完全清晰,那么就审视一下其他的替代形式。例如,可以假设

\placeformula
\startformula
(AB|C) = F[(A|C), (B|C)]
\stopformula

是容许的形式。但是,很容易揭示,这种形式的关系无法满足公设 (\convertnumber{R}{2}) 的那些定性条件。给定 C,命题 $A$ 或许非常可信,命题 $B$ 或许可信,但是 $AB$ 可能依然非常可信或非常不可信。

例如,下一个遇到的人,蓝眼睛,这相当可信,黑头发,这也相当可信;这两样生理特征同时出现于此人身上,也没什么不合理之处。不过,左眼睛是蓝色的,这相当可信,右眼睛是褐色的,也相当可信,但是若它们同时为真,那就极为不可信。如果使用这种形式的公式,那么就没法考虑这些情况。用这种形式的函数,我们的机器人无法像人类那样去作推理,它甚至连定性的推理都做不到。

还有一些其他的可能的形式。尝试所有的可能形式的方法——\quotation{穷举证明}——可像下面这样进行。先引入一些实数

\placeformula
\startformula
u = (AB|C),\quad v = (A|C),\quad w = (B|AC),\quad x = (B|C),\quad y = (A|BC).
\stopformula

如果 $u$ 被表示成 $u$、$v$、$x$、$y$ 中两个或更多个实数的函数,那么有 11 种可能的形式。可以写出每一种可能,将它置于各种极端条件下,如同褐色眼睛与蓝色眼睛那样(抽象描述:$A$ 蕴含了 $B$ 为假\footnote{译注:$A\Rightarrow\itbar{B}$})。其他极端条件有,$A = B$,$A = C$,$C\Rightarrow\itbar{A}$,等。Tribus(1969)作了冗长乏味的分析,发现只有两种可能的形式,在某种极端的情况下,遵循着定性符合常识这一公设,它们分别是 $u = F(x, y)$ 与 $u = F(w, v)$,这正是之前我们推理出来的那两种函数形式。

现在,运用第 \in[plausible-reasoning] 章讨论的量化要求。给定的先验信息的任何变化 $C\rightarrow C'$,以及 $B$ 变得更可信,$A$ 没有变化,

\placeformula
\startformula
B|C' > B|C
\stopformula

\placeformula
\startformula
A|BC' = A|BC
\stopformula

根据常识,$AB$ 只会变得更可信,而不是更不可新:

\placeformula
\startformula
AB|C'\ge AB|C
\stopformula

当且仅当 $A|BC$ 为不可能时,相等关系成立。同样,给定先验信息 $C''$,以及

\placeformula
\startformula
B|C'' = B|C
\stopformula

\placeformula
\startformula
A|BC'' > A|BC
\stopformula

就会有

\placeformula
\startformula
AB|C'' \ge AB|C
\stopformula

当且仅当 $B|C$ 为不可能时,相等关系成立(即使此时未去定义 $A|BC$,$AB|C''$ 依然不可能为真)。此外,函数 $F(x, y)$ 必须是连续的,不然,(\in[basic-function]) 右侧的某个可信性的微小递增可能会导致 $AB|C$ 的大幅递增。

总之,$F(x, y)$ 必须是 x 与 y 的连续的单调递增函数。若假设它可微(并非必须如此;见 (\in[function-eq]) 的讨论),便有

\startsubformulas[mono-req]
\placeformula
\startformula
F_1(x, y) \equiv \frac{\partial F}{\partial x} \ge 0
\stopformula

仅当 $x$ 表示不可能时,上式中的相等关系成立;还有,

\placeformula
\startformula
F_2(x, y) \equiv \frac{\partial F}{\partial y} \ge 0
\stopformula
\stopsubformulas


后文会继续使用这些符号,无论 $F$ 是什么样的函数,$F_i$ 表示 $F$ 的第 $i$ 个参数对应的微分。

接下来,我们动用公设 (\convertnumber{R}{3}a),\quotation{结构上}的一致性。假设想获得 $(ABC|D)$ 的可信性,即三个命题同时为真的可信性,由于布尔运算遵循结合律 $ABC = (AB)C = A(BC)$,因此有两种方式来做此事。若规则具有一致性,一定能从这两种顺序的运算中获得相同的结果。首先,可以将 $BC$ 视为单一的命题,利用 (\in[basic-function]):


\placeformula
\startformula
(ABC|D) = F[(BC|D), (A|BCD)]
\stopformula

然后,对可信性 $(BC|D)$ 应用 (\in[basic-function]),可得:

\startsubformulas[expand-way]
\placeformula[subeq:1]
\startformula
(ABC|D) = F\{F[(C|D), (B|CD)], (A|BCD)\}
\stopformula

不过,也可以在一开始将 $AB$ 视为单一的命题,这样就可以从另一种顺序推出不同的结果:

\placeformula[subeq:2]
\startformula
(ABC|D) = F[(C|D), (AB|CD)] = F\{(C|D), F[(B|CD), (A|BCD)]\}
\stopformula
\stopsubformulas

若我们寻找的这个规则是要用于表示推理的一致性方法,那么 (\in[subeq:1]) 与 (\in[subeq:2]) 必定恒等。在这种情况下,我们的机器人所作的一致性推理,其必要条件可表示为一个函数方程,

\placeformula[function-eq]
\startformula
F[F(x, y), z] = F[x, F(y, z)]
\stopformula

在数学里,这个方程历史悠久,源起于 N. H. Abel (1826) 的研究。Acz\'el (1966) 在他的讲述函数方程的巨著中,贴切地称这个方程为\quotation{关联方程}(Associativity Equation),并列举了 98 份讨论或使用这个方程的参考文献。Acz\'el 在不假设函数可微的条件下,求出了通解 (\in[aczel-general-solution]),见下文。不过,他的书中(Acz\'el, 1987),用了 11 页的篇幅证明此解的存在。在此,我们给出 R. T. Cox (1961) 在假设函数可微的前提下给出的更短的证明;也可参考附录 B 中的讨论。

显然,(\in[function-eq]) 有一个平凡解,$F(x, y) = \text{常数}$。但是这个解违背了单调性要求 (\in[mono-req]),并且毫无用处可言。除非 (\in[function-eq]) 有一个非平凡解,不然,这条路就走不通了;因此,必须寻求最为广义的非平凡解。先定义一些缩写

\placeformula
\startformula
u\equiv F(x, y),\quad v\equiv F(y, z)
\stopformula

现在,依然认为 $(x, y, z)$ 是互不依赖的变量,待求解的函数方程可写为

\placeformula[eq-target]
\startformula
F(x, v) = F(u, z)
\stopformula

分别对 $x$ 与 $y$ 求微分,按照 (\in[mono-req]) 的记法,可得

\placeformula[elim]
\startformula
\startmathalignment
\NC F_1(x, v) \NC = F_1(u, z)F_1(x, y)\NR
\NC F_2(x, v)F_1(y, z) \NC = F_1(u, z)F_2(x, y)\NR
\stopmathalignment
\stopformula

从这两个方程中消除 $F_1(u, z)$,可得

\placeformula[eq-U]
\startformula
G(x, v)F_1(y, z) = G(x, y)
\stopformula

其中,$G(x, y)\equiv F_2(x, y)/F_1(x, y)$。显然,(\in[elim]) 的左部一定是不依赖 $z$。可将 (\in[eq-U]) 等价地写为

\placeformula[eq-V]
\startformula
G(x, v)F_2(y, z) = G(x, y)G(y, z)
\stopformula

用 $U$ 与 $V$ 分别表示 (\in[eq-U]) 与 (\in[eq-V]) 的左部,可以得出 $\partial V/\partial y = \partial U/\partial z$。因而 $G(x, y)G(y, z)$ 必定不依赖 $y$。具有这一性质的最具一般性的函数 $G(x, y)$ 为

\placeformula[general-func]
\startformula
G(x, y) = r\frac{H(x)}{H(y)}
\stopformula

其中,$r$ 为常数,$H(x)$ 为任意函数。基于 $F$ 的单调性,可确定 $G > 0$,因此需要 $r > 0$,至于 $H(x)$ 的正负则无关大体。应用 (\in[general-func]),则 (\in[eq-U]) 与 (\in[eq-V]) 变成:

\placeformula
\startformula
F_1(y, z) = \frac{H(v)}{H(y)} 
\stopformula

\placeformula
\startformula
F_2(y, z) = r\frac{H(v)}{H(z)} 
\stopformula

关系 $dv=dF(y, z)=F_1dy + F_2dz$ 可化为

\placeformula
\startformula
\frac{dv}{H(v)} = \frac{dy}{H(y)} + r\frac{dz}{H(z)}
\stopformula

或者化为积分形式

\placeformula[eq-int]
\startformula
w[F(y, z)] = w(v) = w(y)w'(z)
\stopformula

其中

\placeformula[w-func]
\startformula
w(x) = \text{exp}\left\{\int_{}^{x}\frac{dx}{H(x)}\right\}
\stopformula

积分符号无下界,意味着 $w$ 会有一个任意的倍增因子。但是,将 (\in[eq-target]) 代入函数 $w(\cdot)$ 并应用 (\in[eq-int]),可得 $w(x)w^r(v) = w(u)w^r(z)$;再次应用 (\in[eq-int]),函数方程便可化为

\placeformula[2-25]
\startformula
w(x)w^r(y)[w(z)]^{r^2} = w(x)w^r(y)w^r(z)
\stopformula

若 $r = 1$,便可获得一个非平凡解,最终结果可表示为以下两种形式:

\placeformula
\startformula
w[F(x, y)] = w(x)w(y)
\stopformula

或

\placeformula[aczel-general-solution]
\startformula
F(x, y) = w^{-1}[w(x)w(y)]
\stopformula

因此,逻辑乘法所遵守的结合律与交换律必须体现为以下的函数形式

\placeformula[product-rule]
\startformula
w(AB|C) = w(A|BC)w(B|C) = w(B|AC)w(A|C)
\stopformula

我们将这种形式称为{\bf 乘法规则}。由 (\in[w-func]) 的构造可知,$w(x)$ 必定是个正的连续的单调函数,至于它是递增的还是递减的,这有赖于 $H(x)$ 的符号;目前,它另有深意。

结果得到了 (\in[product-rule]),某种意义上,它是公设 (\convertnumber{R}{3}a) 所述一致性的必要条件。不过,对于任意多个联结的命题,(\in[product-rule]) 显然也能保证这种一致性的存在。例如,用 (\in[expand-way]) 的办法可以将 $(ABCDEFG|H)$ 展开为数目繁多的不同形式;但是,只要 (\in[product-rule]) 成立,这些形式的结果必定相同。

定性要符合常识,这一要求对 $w(x)$ 有着更为严格的限定。例如,根据 (\in[product-rule]) 所给出的形式,假设在给定 $C$ 的情况下 $A$ 为真,那么在由 $C$ 的知识所营造的\quotation{逻辑环境}里,从当且仅当一个为真时另一个也必定为真这一意义来说,命题 $AB$ 与 $B$ 并无区别。基于第 1 节讨论的最基本的公理,同样为真的命题一定具备相同的可信性\footnote{见 1.5 节。}:

\placeformula
\startformula
AB|C = B|C
\stopformula

还有

\placeformula
\startformula
A|BC = A|C
\stopformula

由于给定 $C$ 的时候 $A$ 是确信的(亦即 $C$ 蕴含 $A$),那么给定任何其他不与 $C$ 矛盾的 $B$,$A$ 依然是确信的。在这种情况下,(\in[product-rule]) 约化为

\placeformula
\startformula
w(B|C) = w(A|C)w(B|C)
\stopformula

对于机器人而言,无论 $B$ 有多么可信或不可信,该式必定成立。因此,函数 $w(x)$ 必定具有以下性质

\placeformula
\startformula
w(A|C) = 1\;\text{表示确信性}
\stopformula

现在,在给定 $C$ 的情况下,假设 $A$ 不可信,那么在给定 $C$ 时,命题 $AB$ 也不可信:

\placeformula
\startformula
AB|C = A|C
\stopformula

若给定 $C$ 的情况下,$A$ 不可信(亦即 $C$ 蕴含 $\itbar{A}$),那么给定任何不与 $C$ 矛盾的更充分的信息 $B$,$A$ 依然不可信:

\placeformula
\startformula
A|BC = A|C
\stopformula

在这种情况下,(\in[product-rule]) 可化为

\placeformula[eq-zero]
\startformula
w(A|C) = w(A|C)w(B|C)
\stopformula

无论 $B$ 有多么可信,这个方程必定成立。$w(A|C)$ 只有两个可能的值能够满足这个条件,要么为 $0$,要么为 $+\infty$(排除了 $-\infty$,不然基于连续性,$w(B|C)$ 必须为负值;于是 (\in[eq-zero]) 自相矛盾)。

综上所述,定性要符合常识,这决定了 $w(x)$ 必须是一个正值的连续单调函数。它可能递增,也可能递减。若它递增,取值范围必须从 0 到 1,前者表不可信,后者表确信。若它递减,取值范围必须从 $+\infty$ 到 1,前者表不可信,后者表确信。至于它在这两种区间里具体如何变化,我们所给出的公设则没有对其进行限定。

然而,这两种可能的表示在内涵上并不相同。给定任意函数 $w_1(x)$,令它符合上述标准并且用 $+\infty$ 表示不可信,便可以定义一个新函数 $\displaystyle w_2(x) = \frac{1}{w_1(x)}$,这个函数同样可被接受,它是以 0 来表示不可信。因而,作为一种约定,若我们采纳 $0\le w(x)\le 1$,不失一般性;也就是说,就内涵而言,我们所作的公设里面的所有的一致性皆被包含于这种形式。(读者不妨检验一下,未尝不可选择相反的那种约定,从这一点发展出一套完整的理论,也包括它的全部应用,结果是一样的,只不过方程在形式上有些另类,但内涵是相同的。)

\section{加法规则}

因为现在所考虑的命题为亚里士多德逻辑类型——非真即假,逻辑乘 $A\itbar{A}$ 总是为假,逻辑和 $A + \itbar{A}$ 总是为真。在某些方面,$A$ 为假的可信性一定依赖于它为真的可信性\footnote{译注:$A$ 的可信性依赖于 $\itbar{A}$ 的可信性。}。如果定义 $u\equiv w(A|B)$,$v\equiv w(\itbar{A}|B)$,那么必定存在某种函数关系

\placeformula[2-36]
\startformula
v = S(u)
\stopformula

定性要符合常识。显然,当 $0\le u\le 1$ 时,$S(u)$ 应当是一个连续单调递减函数,其极值 $S(0) = 1$,$S(1) = 0$。不过,具备这些性质的函数未必皆为 $S(u)$,因为 $S(u)$ 必须与 $AB$ 或 $A\itbar{B}$ 的乘法规则

\placeformula[2-37]
\startformula
w(AB|C) = w(A|C)w(B|AC)
\stopformula

\placeformula[2-38]
\startformula
w(A\itbar{B}|C) =  w(A|C)w(\itbar{B}|AC)
\stopalign
\stopformula

保持一致。因而,凭借 (\in[2-36]) 和 (\in[2-38]),可将方程 (\in[2-37]) 变为

\placeformula[2-39]
\startformula
w(AB|C) = w(A|C)S[w(\itbar{B}|AC)] = w(A|C)S\left[\frac{w(A\itbar{B}|C)}{w(A|C)}\right]
\stopformula

我们再用一次交换性:$A$ 和 $B$ 在 $w(AB|C)$ 中是对称的,由一致性\footnote{译注:祈求 (\Roman{3}a)}可得

\placeformula[2-40]
\startformula
w(A|C)S\left[\frac{w(A\itbar{B}|C)}{w(A|C)}\right] = w(B|C)S\left[\frac{w(B\itbar{A}|C)}{w(B|C)}\right]
\stopformula

对于所有命题 $A$、$B$、$C$,这一方程必定成立;特别是,当

\placeformula[2-41]
\startformula
\itbar{B} = AD
\stopformula

时,$D$ 为任何新的命题,(\in[2-40]) 必定成立。不过,如 (\in[1-13]) 所述,有:

\placeformula[2-42]
\startformula
A\itbar{B} = \itbar{B}\quad\quad B\itbar{A} = \itbar{A}
\stopformula

(\in[2-40]) 中的某些项可以写成

\placeformula[2-43]
\startformula
\startmathalignment[n=3]
\NC w(A\itbar{B}|C) \NC = w(\itbar{B}|C) \NC = S[w(B|C)] \NR
\NC w(B\itbar{A}|C) \NC = w(\itbar{A}|C) \NC = S[w(A|C)] \NR
\stopmathalignment
\stopformula

因而,若使用缩写

\placeformula[2-44]
\startformula
x\equiv w(A|C)\,,\quad\quad y\equiv w(B|C)
\stopformula

(\in[2-40])\footnote{译注:原书是 (\in[2-25]),应该有误。} 就变成了函数方程

\placeformula[2-45]
\startformula
xS\left[\frac{S(y)}{x}\right] = yS\left[\frac{S(x)}{y}\right],\quad\quad
\startmathmatrix[align=left]
\NC 0 \le S(y) \le x \NR
\NC 0 \le x \le 1 \NR
\stopmathmatrix
\stopformula

这个方程表示了 $S(x)$ 为了与乘法规则一致而必须具备的一种比例性质。在 $y = 1$ 这种特殊情况下,这个方程便约化为

\placeformula[2-46]
\startformula
S[S(x)] = x
\stopformula

这表明 $S(x)$ 是一个自反函数;$S(x) = S^{-1}(x)$。因此,从 (\in[2-36]) 可以推出 $u = S(v)$。不过,这不过是表示了一个显而意见的事实,$A$ 与 $\itbar{A}$ 成相反关系;这与哪一个命题是用头戴短横线的字母来表示没有关系。之前,我们在 (\in[1-8]) 已经提到了此事;若之前还不够明显,那么现在必须认识到这一点。

下面探究一下 (\in[2-45]) 中的有效域。命题 $D$ 是随意的,因此通过选择不同的 $D$,便可以在

\placeformula[2-47]
\startformula
0 \le w(D|AC) \le 1
\stopformula

时,得到 $w(D|AC)$ 的所有值。但是,$S(y) = w(AD|C) = w(A|C)w(D|AC)$,因此 (\in[2-47]) 恰好能够促成 (\in[2-45]) 所述的 $(0 \le S(y) \le x)$。$x$ 和 $y$ 在该域中是对称的;将二者互换,该域不变。在几何上,该域由 $xy$ 平面上单位正方形 $(0\le x, y\le 1)$ 内的点组成,并且这些点位于曲线 $y = S(x)$ 的上方。

实际上,通过 (\in[2-45]) 描述点对这条曲线的无限逼近,便可确定这条曲线的形状。因为,若设 $y = S(x) + \epsilon$,那么当 $\epsilon \rightarrow 0^{+}$ 时,(\in[2-45]) 中的两边趋向于 $S(1) = 0$,只不过速度不同。因此,一切取决于 $\delta\rightarrow 0$ 的时候 $S(1 - \delta)$ 趋向于 0 的确切途径。为了研究这一现象,我们通过

\placeformula[2-48]
\startformula
\frac{S(x)}{y} = 1 - \text{exp}\{-q\}
\stopformula

定义了一个新的变量 $q(x, y)$。然后,挑选 $\delta = \text{exp}\{-q\}$,通过

\placeformula[2-49]
\startformula
S(1 - \delta) = S(1 - \text{exp}\{-q\}) = \text{exp}\{-J(q)\}
\stopformula

定义函数 $J(q)$,然后探究 $q\rightarrow\infty$ 时 $J(q)$ 的渐近形式。

现在将 $x$ 和 $q$ 视为独立的变量,由 (\in[2-48]) 可得

\placeformula[2-50]
\startformula
S(1 - \delta) = S[S(X)] + \text{exp}\{-q\}S(x)S'[S(x)] + O(\text{exp}\{-2q\})
\stopformula

使用 (\in[2-46]) 以及它的导数 $S'S[S(x)]S'(x) = 1$,上式约化为

\placeformula[2-51]
\startformula
\frac{S(y)}{x} = 1 - \text{exp}\{-(\alpha + q)\} + O(\text{exp}\{-2q\})
\stopformula

其中

\placeformula[2-52]
\startformula
\alpha(x)\equiv\log\left[\frac{-xS'(x)}{S(x)}\right] > 0
\stopformula

使用这些作为替换,函数方程 (\in[2-45]) 变为

\placeformula[2-53]
\startformula
J(q + \alpha) - J(q) = \log\left[\frac{x}{S(x)}\right] + \log(1 - \text{exp}\{-q\}) + O(\text{exp}\{-2q\})
\stopformula

当 $q\rightarrow\infty$ 时,上式的后两项以指数的速度趋向于 0,因此 $J(q)$ 必定是渐近线性的,

\placeformula[2-54]
\startformula
J(q) \sim \alpha + bq + O(\text{exp}\{-2q\})
\stopformula

其斜率

\placeformula[2-55]
\startformula
b = \alpha^{-1}\log\left[\frac{x}{S(x)}\right]
\stopformula

为正值。对于不同的 $x$ 值构成的连续统,及至 $\alpha(x)$ 的值构成的连续统,因为 (\in[2-53]) 必定成立,所以在 $\alpha$ 的变化周期内,(\in[2-54]) 不存在周期项。但是根据定义,$J$ 仅是 $q$ 的函数,因此 (\in[2-55]) 的右侧项一定独立于 $x$。由 (\in[2-52]) 可得

\placeformula[2-56]
\startformula
\frac{x}{S(x)} = \left[\frac{-xS'(x)}{S(x)}\right]^b,\quad\quad 0 < b < \infty
\stopformula

或者,$S(x)$ 必须满足微分方程

\placeformula[2-57]
\startformula
S^{m - 1}{\mathrm{d}}S + x^{m - 1}{\mathrm{d}}x = 0 
\stopformula

其中 $m\equiv 1/b$ 为正数常量。这个方程仅有一个解

\placeformula[2-58]
\startformula
S(x) = (1 - x^m)^{\frac{1}{m}}, \quad\quad
\startmathmatrix[align=left]
\NC 0 \le x \le 1 \NR
\NC 0 < m < \infty \NR
\stopmathmatrix
\stopformula

满足 $S(0) = 1$,并且 (\in[2-58]) 显然是 (\in[2-45]) 的解。

所得结果 (\in[2-58]) 最早由 R. T. Cox (1946) 通过假设 $S(x)$ 二阶可微而推导出来。不需要以 $S(x)$ 可微为前提,Acz\'el (1966) 也推导出了同样的结果。(但是在当前的应用中假设 $S(x)$ 可微对于我们而言无碍,因为若这个函数方程的解为不可微的函数,我们就应该放弃整个理论的构建,原因是定性与常识冲突。)无论如何,(\in[2-58]) 是满足函数方程 (\in[2-45]) 与左边界条件 $S(0) = 1$ 的最具一般性的解;于是我们进而发现它自动满足右边界条件 $S(1) = 0$。

由于我们对函数方程 (\in[2-45]) 的推导用了针对 $B$ 的特殊选择 (\in[2-41]),至此我们仅仅证明了 (\in[2-58]) 是一般性的一致性要求 (\in[2-40] 的必要条件。为了检验其充分性,将 (\in[2-58]) 代入 (\in[2-40]),可得

\placeformula[2-59]
\startformula
w^m(A|C) - w^m(A\itbar{B}|C) = w^m(B|C) - w^m(B\itbar{A}|C)
\stopformula

通过 (\in[2-28]) 与 (\in[2-38]) 很容易验证上式成立。因而,(\in[2-58]) 保证 $S(x)$ 满足 (\in[2-40]) 意义上的一致性的充分必要条件。

至此,可对我们所得的结果作一些总结。逻辑乘法的关联性要求可信性 $x = A|B$ 的某种单调函数 $w(x)$ 必须遵守乘法规则 (\in[2-28])。我们的结果 (\in[2-58]) 描述了 $w(x)$ 也必须遵守加法规则:

\placeformula[2-60]
\startformula
w^m(A|B) + w^m(\itbar{A}|B) = 1
\stopformula

$m$ 为正数。当然,乘法规则本身可被等价地写成

\placeformula[2-61]
\startformula
w^m(AB|C) = w^m(A|C)w^m(B|AC) = w^m(B|C)w^m(A|BC)
\stopformula

实际上,$m$ 值可以忽略,因为无论其取值为何,皆能定义

\placeformula[2-62]
\startformula
p(x) \equiv w^m(x)
\stopformula

这样,乘法与加法规则的形式就变为

\placeformula[2-63]
\startformula
p(AB|C) = p(A|C)P(B|AC) = p(B|C)p(A|BC)
\stopformula

\placeformula[2-64]
\startformula
p(A|B) + p(\itbar{A}|B) = 1
\stopformula

事实上,这样做不失一般性,因为我们对 $w(x)$ 仅有的要求是 $w(x)$ 是一个连续单调递增函数,$w = 0$ 表示不可能,$w = 1$ 表示确定。如果 $w(x)$ 满足这一要求,那么 $w^m(x)$ 也必定满足这一要求,$0 < m < \infty$。因此,声称我们能够使用不同的 $m$ 值并不会给我们带来任何原本在 $w(x)$ 上所没有的自由。我们的祈求所容许的所有可能性均被包含在 (\in[2-63]) 和 (\in[2-64]) 之中了,其中 $p(x)$ 为任意的连续单调递增函数且 $0\le p(x)\le 1$。

对于可信推断而言,要形成一个完备的规则集,用于确定由 $\{A_1,\cdots,A_n\}$ 构成的任意逻辑函数 $f(A_1,\cdots,A_n)$ 的可信性,还需要更多的关系吗?凭借乘法规则 (\in[2-63]) 与加法规则 (\in[2-64]),我们有了计算合取 $AB$ 与取反 $\itbar{A}$ 的可信性的公式。然而,我们已经注意到了,在 (\in[1-23]) 的讨论中,合取与取反可以构成运算完备集,通过它们可以构造所有的逻辑函数。

因而,我们推测基本规则的探索活动已经完成;通过反复应用乘法规则与加法规则,在布尔代数中由 $\{A_1,\cdots,A_n\}$ 产生的任何命题的可信性皆能计算出来。

为了验证这一点,我们首先为逻辑和 $A + B$ 寻找一个公式。反复应用乘法规则与加法规则,可得

\placeformula[2-65]
\startformula
\startmathmatrix[align={middle, middle, left}]
\NC p(A+B|C) \NC = \NC 1 - P(\itbar{A}\itbar{B}|C) = 1 - p(\itbar{A}|C)p(\itbar{B}|\itbar{A}C)\NR
\NC \NC = \NC 1 - p(\itbar{A}|C)[1 - p(B|\itbar{A}C)] = p(A|C) + p(\itbar{A}B|C)\NR
\NC \NC = \NC p(A|C) + p(B|C)p(\itbar{A}|BC) = p(A|C) + p(B|C)[1 - p(A|BC)]\NR
\stopmathmatrix
\stopformula

最终可得

\placeformula[2-66]
\startformula
p(A + B | C) = p(A|C) + p(B|C) - p(AB|C)
\stopformula

这是最实用的通用加法法则。显然,在 $B = \itbar{A}$ 时,基本加法法则 (\in[2-64]) 是 (\in[2-66]) 的特例。

\startframedtext[width=broad]
\exercise{} 能够像 (\in[2-66]) 那样基于乘法与加法规则为 $p(C|A + B)$ 找到通用的公式吗?如果能,就把它推导出来,否则就解释一下为什么不能。
\stopframedtext

\startframedtext[width=broad]
\exercise{} 假设存在命题集合 $\{A_1,\cdots,A_n\}$ 在 $X$ 上相互独立:$p(A_iA_j|X) = p(A_i|X)\delta_{ij}$。证明 $p(C|(A_1 + \cdots + A_n)X)$ 为各个 $P(C|A_iX)$ 的加权平均:

\placeformula[2-67]
\startformula
p(C|(A_1 + \cdots + A_n)X) = p(C|A_1X + \cdots + A_nX) = \frac{\sum_i p(A_i|X)p(C|A_iX)}{\sum_i p(A_i|X)}
\stopformula
\stopframedtext

为了扩展 (\in[2-66]),在 (\in[1-17]) 之后,我们注意到除了没什么价值的自相矛盾的逻辑函数之外,任何一个逻辑函数皆能表示为析取范式,即作为诸如 (\in[1-17]) 所示的基本合取运算的逻辑和。现在,任何一个基本的合取运算的可信性 $\{Q_i,\; 1\le i \le 2^n\}$ 可由乘法规则的反复应用而确定;然后反复应用 (\in[2-66]) 便可确定 $Q_i$ 所形成的任何逻辑和的可信性。实际上,这些合取运算是相互独立的,因此我们将会发现(见 (\in[2-85]))这种逻辑和可约化为 $(2^n - 1)$ 个项构成的简单求和运算 $\sum_ip(Q_i|C)$。

因此,如同合取与取反这两种运算可构成演绎逻辑的完备集那样,上述的乘法与加法规则在以下意义中可构成可信推断的完备集。只要背景信息足以确定基本合取运算的可信性,我们的规则就足以确定布尔代数中由 $\{A_1\cdots A_n\}$ 产生的任何命题的可信性。因而,在 $n = 4$ 的情况下,需要 $2^4 = 16$ 个基本合取运算,于是我们的规则可以确定布尔代数中 $2^{16} = 65\,536$ 个命题中的每一个的可信性。

不过,在实际应用中我们用不着确定这么多;若背景信息足以确定少量的基本合取运算的可信性,这可能对于我们所关心的布尔代数中一小部分命题足够用了。

\section{定性性质}

现在来看基于 (\in[2-63]) 与 (\in[2-64]) 的理论、演绎逻辑的理论以及我们在第 1 章给出的几个定性的三段论之间存在怎样的联系。首先,在极限 $p(A|B)\rightarrow 0$ 或者 $p(A|B) \rightarrow 1$ 存在的情况下,显然加法规则 (\in[2-64]) 表达的是亚里士多德逻辑的先决条件:若 $A$ 为真,则 $\itbar{A}$ 必定为假,等等。

的确,两条强三段论 (\in[syllogism-1]) 和 (\in[syllogism-2]) 以及由它们发展出来的弱三段论所构成的逻辑,以及这些逻辑所构成的层出不穷的结果,它们的大前提都可以使用蕴含符号 (\in[implication]) 来描述:

\placeformula[2-68]
\startformula
\syllogism{$A\Rightarrow B$}{$A$ 为真}{$B$ 为真}\quad\quad\syllogism{$A\Rightarrow B$}{$B$ 为假}{$A$ 为假}\NR
\stopformula

若用 $C$ 表示它们的大前提:

\placeformula[2-69]
\startformula
C\equiv A\Rightarrow B
\stopformula

则这些三段论便会以下面的形式分别与乘法规则 (\in[2-63]) 相对应:

\placeformula[2-70]
\startformula
p(B|AC) = \frac{p(AB|C)}{p(A|C)}\quad\quad p(A|\itbar{B}C) = \frac{p(A\itbar{B}|C)}{p(\itbar{B}|C)}
\stopformula

但是,根据 (\in[2-68]),有 $p(AB|C) = p(A|C)$ 和 $p(A\itbar{B}|C) = 0$,因此 (\in[2-70]) 可约化为

\placeformula[2-71]
\startformula
p(B|AC) = 1\quad\quad p(A|\itbar{B}C) = 0
\stopformula

与三段论 (\in[2-68]) 所述一致。因而这种联系很简单:{\bf 亚里士多德的演绎逻辑是可信推理规则的极限形式,意味着机器人对它所得到的结论越来越确定。}

但是我们的规则还有演绎逻辑不具备的东西:若三段论 (\in[weak-1]) 与 (\in[weak-2]) 的量化形式。为了证明原始定性描述总是遵循当前规则,需要注意第一个弱三段论

\placeformula[2-72]
\startformula
\syllogism{$A\Rightarrow B$}{$B$ 为真}{所以,$A$ 更可信}
\stopformula

以下面的形式

\placeformula[2-73]
\startformula
p(A|BC) = p(A|C)\frac{p(B|AC)}{p(B|C)}
\stopformula

与乘法规则 (\in[2-63]) 相对应。但是,根据 (\in[2-68]),$p(B|AC) = 1$,且因 $p(B|C)\le 1$,所以可得

\placeformula[2-74]
\startformula
p(A|BC) \ge p(A|C)
\stopformula

结果与弱三段论相符。同理,弱三段论 (\in[weak-2])

\placeformula[2-75]
\startformula
\syllogism{$A\Rightarrow B$}{$A$ 为假}{所以,$B$ 更不可信}
\stopformula

以下面的形式

\placeformula[2-76]
\startformula
p(B|\itbar{A}C) = p(B|C)\frac{p(\itbar{B}|AC)}{p(B|C)}
\stopformula

与乘法规则相对应。但是根据 (\in[2-74]),有 $p(A\itbar{A}B)\le p(\itbar{A}|C)$,因此由 (\in[2-76]) 可得

\placeformula[2-77]
\startformula
p(B|\itbar{A}C) \le p(B|C)
\stopformula

结果与上述弱三段论相符。

最后,那个警察的弱三段论 (\in[weak-3]),在抽象描述时似乎非常弱,也以 (\in[2-73]) 的形式被我们的乘法规则涵盖在内。现在让 $C$ 表示背景信息(为了不横生枝节,在 (\in[weak-3]) 中对此未作强调),大前提\quotation{若 $A$ 为真,则 $B$ 更可信}可表示为

\placeformula[2-78]
\startformula
p(B|AC) > p(B|C)
\stopformula

由 (\in[2-73]) 可得

\placeformula[2-79]
\startformula
p(A|BC) > p(A|C)
\stopformula

结果与这个三段论相符。

我们所得到的不仅仅是 (\in[2-79]) 这样的定性描述。在第 1 节里,有个未回答的疑问:什么因素决定了证据 $B$ 是将 $A$ 几乎提升到了确信无疑的程度,还是对其可信性有负面影响?来自 (\in[2-73]) 的答案是,因为 $p(B|AC)$ 不能大于 1,仅当 $p(B|C)$ 很小的时候,$A$ 的可信性才有机会得以大幅递增。之所以看到那个绅士的行为($B$)使其罪行($A$)在事实上成立,原因在于他的行为在背景信息上只能如此解释;警察在此前未曾见识过无罪之人会作出那种行为。另一方面,若已知 $A$ 为真只会让 $B$ 的可信性有着可忽略的递增,则看到 $B$ 也只会让 $A$ 的可信性有着可忽略的递增。

我们可以给出这种类型的更多比较;实际上,这些规则与常识之间完备的定性对应已经被许多人注意并证明了,诸如 Keynes(1921)、Jeffreys(1939)、P\'olya(1945,1954)、R. T. Cox(1961)、Tribus(1969)、de Finetti(1947a,b)以及 Rosenkrantz(1977)等人的工作。本书前言与第 1 章主要介绍了 P\'olya 的工作,刚才我们所讲的主要是 Cox 的工作。然而,我们的目标是推进量化的应用;因此我们要回到这个主题上来。

\section{数值}

至此,我们已经发现了最为一般的一致性规则,通过这些规则我们的机器人可以操控可信性,并且必须为可信性与实数建立关联,因此它的大脑可以通过执行一些确定的物理过程进行运算。尽管熟悉这些规则的形式化表示以及它们的定性属性鼓舞了我们,但是两个明显的情况证明我们对这个机器人大脑的设计任务尚未完成。

首先,尽管规则 (\in[2-63]) 和 (\in[2-64]) 在不同命题的可信性如何必然相互关联方面给出了一些限定,但是显然我们还没有发现任何{\bf 唯一的}规则,而是无穷个规则,我们的机器人可以通过它们中的任意一个进行可信推理。与单调函数 $p(x)$ 的每个不同的选择相应,似乎存在一个内容不同的规则集。

再者,迄今为止,没有什么告诉我们在问题的一开始应当如何为可信性赋以真实的数值,以使这个机器人开始它的计算。这个机器人如何将背景信息的初始编码转化为确定的可信性数值?为此,我们必须求助于尚未使用的\quotation{界面}祈求 (\in[desiderata-3c]) 的 (\Roman{3}b) 和 (\Roman{3}b)。

下面的分析以有趣且匪夷所思的方式回答了这两个问题。我们先来考虑 $\{A_1,A_2,A_3\}$ 这三个命题中至少一个为真的可信性,即 $(A_1 + A_2 + A_3|B)$ 的可信性。通过对扩展的加法规则 (\in[2-66]) 的两次应用便可确定这个命题的可信性。第一次可得

\placeformula[2-80]
\startformula
p(A_1 + A_2 + A_3|B) = p(A_1 + A_2|B) + p(A_3|B) - p(A_1A_3 + A_2A_3|B)
\stopformula

其中首先将 $(A_1 + A_2)$ 视为一个单一命题,然后使用逻辑关系

\placeformula[2-81]
\startformula
(A_1 + A_2)A_3 = A_1A_3 + A_2A_3
\stopformula

再次应用 (\in[2-66]),可得下面的 7 项:

\placeformula[2-82]
\startformula
\startmathmatrix[align={middle, middle, left}]
\NC p(A_1 + A_2 + A_3|B) \NC = \NC p(A_1|B) + p(A_2|B) + p(A_3|B)\NR
\NC \NC \NC -p(A_1A_2|B) - p(A_2A_3|B) - p(A_3A_2|B)\NR
\NC \NC \NC +p(A_1A_2A_3|B)\NR
\stopmathmatrix
\stopformula

现在假设这些命题相互独立;亦即,证据 $B$ 暗示着这些命题中任意两个能够同时为真:

\placeformula[2-83]
\startformula
p(A_iA_j|B) = p(A_i|B)\delta_{ij}
\stopformula

这样 (\in[2-82]) 的后四项就被消除了,结果为

\placeformula[2-84]
\startformula
p(A_1 + A_2 + A_3|B) = p(A_1|B) + p(A_2|B) + p(A_3|B)
\stopformula

增加更多的命题 $A_4$,$A_5$ 等,通过归纳法很容易证明,对于 $n$ 个相互独立的命题 $\{A_1,\cdots,A_n\}$,可将 (\in[2-84]) 推广为

\placeformula[2-85]
\startformula
p(A_1 + \cdots + A_m|B) = \sum_{i = 1}^{m}p(A_i|B)\quad 1\le m \le n
\stopformula

从现在开始,我们会不断使用这个规则。

传统的阐述是将 (\in[2-85]) 作为一个基本的公理而引入的,由此可见,定义公理有多么任性。上述途径证明了这个规则是简单的一致性定性条件的演绎。将 (\in[2-85]) 视为基本关系,这一观点是我们要致力避免的(见本章的评论部分)。

现在假设命题集合 $\{A_1,\cdots,A_n\}$ 不仅相互独立,而且详尽;亦即,背景信息 $B$ 保证有且仅有一个命题必定为真。在这种情况下,$m = n$,(\in[2-85]) 的结果必定 1:

\placeformula[2-86]
\startformula
\sum_{i = 1}^{n}p(A_i|B) = 1
\stopformula

只凭这一点还不足以确定各个数值 $p(A_i|B)$。许多不同的选择都合适,这取决于信息 $B$ 的更深层次的细节。通过对 $B$ 的逻辑分析来确定 $p(A_i|B)$ 也并非易事。实际上,这是个开放的问题,因为可能被 $B$ 包含的复杂信息的变体无限多;故而将信息转化为 $p(A_i|B)$ 的数值这一过程所涉及的数学问题也无限多。将会看到,这是当前的研究中最重要的问题之一;对于将信息 $B$ 转化为 $p(A_i|B)$ 的数值,我们所能发现的每个新的原理都会开启这个理论的新的一类有价值的应用。