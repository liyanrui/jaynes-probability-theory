\chapter[quantitative-rules]{量化规则}

\startdictum
概率论,不过是生活约化而成的计算。
  
\rightaligned{Laplace,1819}
\stopdictum

\indentation 问题已被形式化了,这是我们所作的公设在数学上的必然结果。这些公设可粗略描述如下:

\startitemize[R]
\item 命题的可信性由实数表示;
\item 定性符合常识;
\item 一致性。
\stopitemize

\noindent 本章仅基于这些公设推导与推断相关的量化规则。所得结果,曾经有着一段漫长、复杂又令人难以置信的历史。这段历史充满着广义科学方法论方面的教训(见某些章尾部的评注部分)。

\section{乘法规则}

我们首先为逻辑乘 $AB$ 的可信性分别与 $A$ 与 $B$ 的可信性之间的关系找出一个一致性的规则。我们在意的其实是 $AB|C$。由于推理过程本身有些微妙,这需要我们从一些不同的视角去审视。

先考虑将 $AB$ 为真的裁决打碎为对 $A$ 与 $B$ 所作的更为基本的裁决。机器人可以

\startitemize[n]
\item 裁决 $B$ 为真;\hfill $(B|C)$
\item 认可 $B$ 为真,裁决 $A$ 为真。\hfill $(A|BC)$
\stopitemize

\noindent 或者,等同地,

\startitemize[n]
\item 裁决 $A$ 为真;\hfill $(A|C)$
\item 认可 $A$ 为真,裁决 $B$ 为真。\hfill $(B|AC)$
\stopitemize

\noindent 上面,在每种情况中,我们给出了每一步骤相应的可信性。

现在,解释一下第 1 个过程。为了命题 $AB$ 为真,$B$ 必须为真。因而,必须考虑 $B|C$ 的可信性。此外,若 $B$ 为真,则需要进一步保证 $A$ 为真,因此还需要考虑 $A|BC$ 的可信性。但是若 $B$ 为假,则无需裁决 $A$,亦即无需裁决 $A|\itbar{B}C$,便可知晓 $AB$ 为假;若机器人先推理 $B$,那么,$A$ 的可信性仅在 $B$ 为真时才值得考虑。因此,机器人在有了 $B|C$ 与 $A|BC$ 的情况下,就不需要 $A|C$ 了,因为 $A|C$ 并未给它能带来更多的信息。

类似地,$A|B$ 与 $B|A$ 是不需要的;无论 $A$ 与 $B$ 在缺乏 $C$ 的情况下多么可信,都与机器人在知道 $C$ 为真的情况下所作的裁决无关。例如,若机器人知晓地球是圆的,那么在裁决与现在的宇宙学相关的问题时,它就可以忽略那些在它尚不知地球是圆的之时需要考虑的观点。

由于逻辑乘法运算遵循交换律,$AB = BA$,因此上文中的 $A$ 与 $B$ 毫无疑问,可以互换;亦即 $A|C$ 与 $B|AC$ 的知识也能用于确定 $AB|C = BA|C$。对于 $AB$,机器人必定能从这两个过程中得到相同的可信性,这是我们的一致性条件之一,即公设 (\convertnumber{R}{3}a)。

可以用更明确的形式来描述。$(AB|C)$ 可以是 $B|C$ 与 $A|BC$ 的某个函数:

\placeformula[basic-function]
\startformula
(AB|C) = F[(B|C), (A|BC)]
\stopformula

现在,如果上述推理尚不完全清晰,那么就审视一下其他的替代形式。例如,可以假设

\placeformula
\startformula
(AB|C) = F[(A|C), (B|C)]
\stopformula

是容许的形式。但是,很容易揭示,这种形式的关系无法满足公设 (\convertnumber{R}{2}) 的那些定性条件。给定 C,命题 $A$ 或许非常可信,命题 $B$ 或许可信,但是 $AB$ 可能依然非常可信或非常不可信。

例如,下一个遇到的人,蓝眼睛,这相当可信,黑头发,这也相当可信;这两样生理特征同时出现于此人身上,也没什么不合理之处。不过,左眼睛是蓝色的,这相当可信,右眼睛是褐色的,也相当可信,但是若它们同时为真,那就极为不可信。如果使用这种形式的公式,那么就没法考虑这些情况。用这种形式的函数,我们的机器人无法像人类那样去作推理,它甚至连定性的推理都做不到。

还有一些其他的可能的形式。尝试所有的可能形式的方法——\quotation{穷举证明}——可像下面这样进行。先引入一些实数

\placeformula
\startformula
u = (AB|C),\quad v = (A|C),\quad w = (B|AC),\quad x = (B|C),\quad y = (A|BC).
\stopformula

如果 $u$ 被表示成 $u$、$v$、$x$、$y$ 中两个或更多个实数的函数,那么有 11 种可能的形式。可以写出每一种可能,将它置于各种极端条件下,如同褐色眼睛与蓝色眼睛那样(抽象描述:$A$ 蕴含了 $B$ 为假\footnote{译注:$A\Rightarrow\itbar{B}$})。其他极端条件有,$A = B$,$A = C$,$C\Rightarrow\itbar{A}$,等。Tribus(1969)作了冗长乏味的分析,发现只有两种可能的形式,在某种极端的情况下,遵循着定性符合常识这一公设,它们分别是 $u = F(x, y)$ 与 $u = F(w, v)$,这正是之前我们推理出来的那两种函数形式。

现在,运用第 \in[plausible-reasoning] 章讨论的量化要求。给定的先验信息的任何变化 $C\rightarrow C'$,以及 $B$ 变得更可信,$A$ 没有变化,

\placeformula
\startformula
B|C' > B|C
\stopformula

\placeformula
\startformula
A|BC' = A|BC
\stopformula

根据常识,$AB$ 只会变得更可信,而不是更不可新:

\placeformula
\startformula
AB|C'\ge AB|C
\stopformula

当且仅当 $A|BC$ 为不可能时,相等关系成立。同样,给定先验信息 $C''$,以及

\placeformula
\startformula
B|C'' = B|C
\stopformula

\placeformula
\startformula
A|BC'' > A|BC
\stopformula

就会有

\placeformula
\startformula
AB|C'' \ge AB|C
\stopformula

当且仅当 $B|C$ 为不可能时,相等关系成立(即使此时未去定义 $A|BC$,$AB|C''$ 依然不可能为真)。此外,函数 $F(x, y)$ 必须是连续的,不然,(\in[basic-function]) 右侧的某个可信性的微小递增可能会导致 $AB|C$ 的大幅递增。

总之,$F(x, y)$ 必须是 x 与 y 的连续的单调递增函数。若假设它可微(并非必须如此;见 (\in[function-eq]) 的讨论),便有

\startsubformulas[mono-req]
\placeformula
\startformula
F_1(x, y) \equiv \frac{\partial F}{\partial x} \ge 0
\stopformula

仅当 $x$ 表示不可能时,上式中的相等关系成立;还有,

\placeformula
\startformula
F_2(x, y) \equiv \frac{\partial F}{\partial y} \ge 0
\stopformula
\stopsubformulas


后文会继续使用这些符号,无论 $F$ 是什么样的函数,$F_i$ 表示 $F$ 的第 $i$ 个参数对应的微分。

接下来,我们动用公设 (\convertnumber{R}{3}a),\quotation{结构上}的一致性。假设想获得 $(ABC|D)$ 的可信性,即三个命题同时为真的可信性,由于布尔运算遵循结合律 $ABC = (AB)C = A(BC)$,因此有两种方式来做此事。若规则具有一致性,一定能从这两种顺序的运算中获得相同的结果。首先,可以将 $BC$ 视为单一的命题,利用 (\in[basic-function]):


\placeformula
\startformula
(ABC|D) = F[(BC|D), (A|BCD)]
\stopformula

然后,对可信性 $(BC|D)$ 应用 (\in[basic-function]),可得:

\startsubformulas[expand-way]
\placeformula[subeq:1]
\startformula
(ABC|D) = F\{F[(C|D), (B|CD)], (A|BCD)\}
\stopformula

不过,也可以在一开始将 $AB$ 视为单一的命题,这样就可以从另一种顺序推出不同的结果:

\placeformula[subeq:2]
\startformula
(ABC|D) = F[(C|D), (AB|CD)] = F\{(C|D), F[(B|CD), (A|BCD)]\}
\stopformula
\stopsubformulas

若我们寻找的这个规则是要用于表示推理的一致性方法,那么 (\in[subeq:1]) 与 (\in[subeq:2]) 必定恒等。在这种情况下,我们的机器人所作的一致性推理,其必要条件可表示为一个函数方程,

\placeformula[function-eq]
\startformula
F[F(x, y), z] = F[x, F(y, z)]
\stopformula

在数学里,这个方程历史悠久,源起于 N. H. Abel (1826) 的研究。Acz\'el (1966) 在他的讲述函数方程的巨著中,贴切地称这个方程为\quotation{关联方程}(Associativity Equation),并列举了 98 份讨论或使用这个方程的参考文献。Acz\'el 在不假设函数可微的条件下,求出了通解 (\in[aczel-general-solution]),见下文。不过,他的书中(Acz\'el, 1987),用了 11 页的篇幅证明此解的存在。在此,我们给出 R. T. Cox (1961) 在假设函数可微的前提下给出的更短的证明;也可参考附录 B 中的讨论。

显然,(\in[function-eq]) 有一个平凡解,$F(x, y) = \text{常数}$。但是这个解违背了单调性要求 (\in[mono-req]),并且毫无用处可言。除非 (\in[function-eq]) 有一个非平凡解,不然,这条路就走不通了;因此,必须寻求最为广义的非平凡解。先定义一些缩写

\placeformula
\startformula
u\equiv F(x, y),\quad v\equiv F(y, z)
\stopformula

现在,依然认为 $(x, y, z)$ 是互不依赖的变量,待求解的函数方程可写为

\placeformula[eq-target]
\startformula
F(x, v) = F(u, z)
\stopformula

分别对 $x$ 与 $y$ 求微分,按照 (\in[mono-req]) 的记法,可得

\placeformula[elim]
\startformula
\startmathalignment
\NC F_1(x, v) \NC = F_1(u, z)F_1(x, y)\NR
\NC F_2(x, v)F_1(y, z) \NC = F_1(u, z)F_2(x, y)\NR
\stopmathalignment
\stopformula

从这两个方程中消除 $F_1(u, z)$,可得

\placeformula[eq-U]
\startformula
G(x, v)F_1(y, z) = G(x, y)
\stopformula

其中,$G(x, y)\equiv F_2(x, y)/F_1(x, y)$。显然,(\in[elim]) 的左部一定是不依赖 $z$。可将 (\in[eq-U]) 等价地写为

\placeformula[eq-V]
\startformula
G(x, v)F_2(y, z) = G(x, y)G(y, z)
\stopformula

用 $U$ 与 $V$ 分别表示 (\in[eq-U]) 与 (\in[eq-V]) 的左部,可以得出 $\partial V/\partial y = \partial U/\partial z$。因而 $G(x, y)G(y, z)$ 必定不依赖 $y$。具有这一性质的最具一般性的函数 $G(x, y)$ 为

\placeformula[general-func]
\startformula
G(x, y) = r\frac{H(x)}{H(y)}
\stopformula

其中,$r$ 为常数,$H(x)$ 为任意函数。基于 $F$ 的单调性,可确定 $G > 0$,因此需要 $r > 0$,至于 $H(x)$ 的正负则无关大体。应用 (\in[general-func]),则 (\in[eq-U]) 与 (\in[eq-V]) 变成:

\placeformula
\startformula
F_1(y, z) = \frac{H(v)}{H(y)} 
\stopformula

\placeformula
\startformula
F_2(y, z) = r\frac{H(v)}{H(z)} 
\stopformula

关系 $dv=dF(y, z)=F_1dy + F_2dz$ 可化为

\placeformula
\startformula
\frac{dv}{H(v)} = \frac{dy}{H(y)} + r\frac{dz}{H(z)}
\stopformula

或者化为积分形式

\placeformula[eq-int]
\startformula
w[F(y, z)] = w(v) = w(y)w'(z)
\stopformula

其中

\placeformula[w-func]
\startformula
w(x) = \text{exp}\left\{\int_{}^{x}\frac{dx}{H(x)}\right\}
\stopformula

积分符号无下界,意味着 $w$ 会有一个任意的倍增因子。但是,将 (\in[eq-target]) 代入函数 $w(\cdot)$ 并应用 (\in[eq-int]),可得 $w(x)w^r(v) = w(u)w^r(z)$;再次应用 (\in[eq-int]),函数方程便可化为

\placeformula[2-25]
\startformula
w(x)w^r(y)[w(z)]^{r^2} = w(x)w^r(y)w^r(z)
\stopformula

若 $r = 1$,便可获得一个非平凡解,最终结果可表示为以下两种形式:

\placeformula
\startformula
w[F(x, y)] = w(x)w(y)
\stopformula

或

\placeformula[aczel-general-solution]
\startformula
F(x, y) = w^{-1}[w(x)w(y)]
\stopformula

因此,逻辑乘法所遵守的结合律与交换律必须体现为以下的函数形式

\placeformula[product-rule]
\startformula
w(AB|C) = w(A|BC)w(B|C) = w(B|AC)w(A|C)
\stopformula

我们将这种形式称为{\bf 乘法规则}。由 (\in[w-func]) 的构造可知,$w(x)$ 必定是个正的连续的单调函数,至于它是递增的还是递减的,这有赖于 $H(x)$ 的符号;目前,它另有深意。

结果得到了 (\in[product-rule]),某种意义上,它是公设 (\convertnumber{R}{3}a) 所述一致性的必要条件。不过,对于任意多个联结的命题,(\in[product-rule]) 显然也能保证这种一致性的存在。例如,用 (\in[expand-way]) 的办法可以将 $(ABCDEFG|H)$ 展开为数目繁多的不同形式;但是,只要 (\in[product-rule]) 成立,这些形式的结果必定相同。

定性要符合常识,这一要求对 $w(x)$ 有着更为严格的限定。例如,根据 (\in[product-rule]) 所给出的形式,假设在给定 $C$ 的情况下 $A$ 为真,那么在由 $C$ 的知识所营造的\quotation{逻辑环境}里,从当且仅当一个为真时另一个也必定为真这一意义来说,命题 $AB$ 与 $B$ 并无区别。基于第 1 节讨论的最基本的公理,同样为真的命题一定具备相同的可信性\footnote{见 1.5 节。}:

\placeformula
\startformula
AB|C = B|C
\stopformula

还有

\placeformula
\startformula
A|BC = A|C
\stopformula

由于给定 $C$ 的时候 $A$ 是确信的(亦即 $C$ 蕴含 $A$),那么给定任何其他不与 $C$ 矛盾的 $B$,$A$ 依然是确信的。在这种情况下,(\in[product-rule]) 约化为

\placeformula
\startformula
w(B|C) = w(A|C)w(B|C)
\stopformula

对于机器人而言,无论 $B$ 有多么可信或不可信,该式必定成立。因此,函数 $w(x)$ 必定具有以下性质

\placeformula
\startformula
w(A|C) = 1\;\text{表示确信性}
\stopformula

现在,在给定 $C$ 的情况下,假设 $A$ 不可信,那么在给定 $C$ 时,命题 $AB$ 也不可信:

\placeformula
\startformula
AB|C = A|C
\stopformula

若给定 $C$ 的情况下,$A$ 不可信(亦即 $C$ 蕴含 $\itbar{A}$),那么给定任何不与 $C$ 矛盾的更充分的信息 $B$,$A$ 依然不可信:

\placeformula
\startformula
A|BC = A|C
\stopformula

在这种情况下,(\in[product-rule]) 可化为

\placeformula[eq-zero]
\startformula
w(A|C) = w(A|C)w(B|C)
\stopformula

无论 $B$ 有多么可信,这个方程必定成立。$w(A|C)$ 只有两个可能的值能够满足这个条件,要么为 $0$,要么为 $+\infty$(排除了 $-\infty$,不然基于连续性,$w(B|C)$ 必须为负值;于是 (\in[eq-zero]) 自相矛盾)。

综上所述,定性要符合常识,这决定了 $w(x)$ 必须是一个正值的连续单调函数。它可能递增,也可能递减。若它递增,取值范围必须从 0 到 1,前者表不可信,后者表确信。若它递减,取值范围必须从 $+\infty$ 到 1,前者表不可信,后者表确信。至于它在这两种区间里具体如何变化,我们所给出的公设则没有对其进行限定。

然而,这两种可能的表示在内涵上并不相同。给定任意函数 $w_1(x)$,令它符合上述标准并且用 $+\infty$ 表示不可信,便可以定义一个新函数 $\displaystyle w_2(x) = \frac{1}{w_1(x)}$,这个函数同样可被接受,它是以 0 来表示不可信。因而,作为一种约定,若我们采纳 $0\le w(x)\le 1$,不失一般性;也就是说,就内涵而言,我们所作的公设里面的所有的一致性皆被包含于这种形式。(读者不妨检验一下,未尝不可选择相反的那种约定,从这一点发展出一套完整的理论,也包括它的全部应用,结果是一样的,只不过方程在形式上有些另类,但内涵是相同的。)

\section{加法规则}

由现在所考虑的命题为亚里士多德逻辑类型——非真即假,逻辑乘 $A\itbar{A}$ 总是为假,逻辑和 $A + \itbar{A}$ 总是为真。在某些方面,$A$ 为假的可信性一定依赖于它为真的可信性。\footnote{译注:$A$ 的可信性依赖于 $\itbar{A}$ 的可信性。} 如果定义 $u\equiv w(A|B)$,$v\equiv w(\itbar{A}|B)$,那么必定存在某种函数关系

\placeformula[2-36]
\startformula
v = S(u)
\stopformula

显然,定性符合尝试,这一祈求要求,在 $0\le u\le 1$ 时,$S(u)$ 是一个连续单调递减函数,其极值 $S(0) = 1$,$S(1) = 0$。不过,它不可能是这些性质的任意函数,因为它必须一个事实保持一致,该事实是 $AB$ 或 $A\itbar{B}$ 的乘法规则可写为:

\placeformula[2-37]
\startformula
w(AB|C) = w(A|C)w(B|AC)
\stopformula

\placeformula[2-38]
\startformula
w(A\itbar{B}|C) =  w(A|C)w(\itbar{B}|AC)
\stopalign
\stopformula

因而,使用 (\in[2-36]) 和 (\in[2-38]),方程 (\in[2-37]) 变成

\placeformula[2-39]
\startformula
w(AB|C) = w(A|C)S[w(\itbar{B}|AC)] = w(A|C)S\left[\frac{w(A\itbar{B}|C)}{w(A|C)}\right]
\stopformula

我们再用一次交换性:$A$ 和 $B$ 在 $w(AB|C)$ 中是对称的,因此一致性\footnote{译注:祈求 (\Roman{3})}要求

\placeformula[2-40]
\startformula
w(A|C)S\left[\frac{w(A\itbar{B}|C)}{w(A|C)}\right] = w(B|C)S\left[\frac{w(B\itbar{A}|C)}{w(B|C)}\right]
\stopformula

对于所有命题 $A$、$B$、$C$,该式一定成立;特别是,当

\placeformula[2-41]
\startformula
\itbar{B} = AD
\stopformula

时,$D$ 为任何新的命题,此时 (\in[2-40]) 一定成立。不过,如 (\in[1-13]) 所述,有:

\placeformula[2-42]
\startformula
A\itbar{B} = \itbar{B}\quad\quad B\itbar{A} = \itbar{A}
\stopformula

(\in[2-40]) 中的某些项可以写成

\placeformula[2-43]
\startformula
\startmathalignment[n=3]
\NC w(A\itbar{B}|C) \NC = w(\itbar{B}|C) \NC = S[w(B|C)] \NR
\NC w(B\itbar{A}|C) \NC = w(\itbar{A}|C) \NC = S[w(A|C)] \NR
\stopmathalignment
\stopformula

因而,若使用缩写

\placeformula[2-44]
\startformula
x\equiv w(A|C)\,,\quad\quad y\equiv w(B|C)
\stopformula

(\in[2-40])\footnote{译注:原书是 (\in[2-25]),应该有误。} 就变成了函数方程

\placeformula[2-45]
\startformula
xS\left[\frac{S(y)}{x}\right] = yS\left[\frac{S(x)}{y}\right],\quad\quad
\startmathmatrix
\NC 0 \le S(y) \le x \NR
\NC 0 \le x \le 1 \NR
\stopmathmatrix
\stopformula

这个方程表示了 $S(x)$ 为了与乘法规则一致而必须具备的一种比例性质。在 $y = 1$ 这种特殊情况下,这个方程便约化为

\placeformula[2-46]
\startformula
S[S(x)] = x
\stopformula

这表明 $S(x)$ 是一个自反函数;$S(x) = S^{-1}(x)$。因此,从 (\in[2-36]) 可以推出 $u = S(v)$。不过,这不过是表示了一个显而意见的事实,$A$ 与 $\itbar{A}$ 成相反关系;这与哪一个命题是用头戴短横线的字母来表示没有关系。之前,我们在 (\in[1-8]) 已经提到了此事;若之前还不够明显,那么现在必须认识到这一点。

下面探究一下 (\in[2-45]) 中的有效域。命题 $D$ 是随意的,因此通过选择不同的 $D$,便可以得到 $w(D|AC)$ 的所有值

\placeformula[2-47]
\startformula
0 \le w(D|AC) \le 1
\stopformula

但是,$S(y) = w(AD|C) = w(A|C)w(D|AC)$,因此 (\in[2-47]) 正是 (\in[2-45]) 所述的 $(0 \le S(y) \le x)$。$x$ 和 $y$ 在这个域里是对称的;将二者互换,这个域不变。在几何上,该域